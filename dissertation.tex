%%%---PREAMBLE---%%%%%%%%%%%%%%%%%%%%%%%%%%%%
\documentclass[oneside,12pt,final]{sty/ucthesis-CA2012}
\pdfoutput=1

%--- Packages ---------------------------------------------------------

\makeatletter
%\let\normalsize\relax
\let\@currsize\normalsize
\makeatother

\usepackage[utf8]{inputenc}
\usepackage{multirow}

\usepackage{graphicx} % Required to insert images
\usepackage{amsmath}
\usepackage{amssymb}
\usepackage{color}

\usepackage{algorithm}
\usepackage[noend]{algpseudocode}

\usepackage{csquotes}
\DeclareMathOperator*{\argmax}{arg\,max}
\DeclareMathOperator*{\argmin}{arg\,min}
\renewcommand{\vec}[1]{\mathbf{#1}}
\usepackage{float}
\usepackage{commath}
\usepackage{booktabs}
\usepackage{hyperref}
% in preamble
\usepackage{array}
\usepackage{fancyhdr}
\usepackage{hyperref}
\usepackage{xspace}
\usepackage{braket}
\usepackage{color}
\usepackage{setspace}

\renewcommand{\vec}[1]{\mathbf{#1}}

\usepackage{tabu}
\usepackage{multirow}


\usepackage{wrapfig}
\usepackage{lipsum}

\usepackage{breqn}
\usepackage{pdfpages}

\usepackage[english]{babel}

\usepackage{caption}
\usepackage{subcaption}





%---New Definitions and Commands------------------------------------------------------
\def\p{\partial}
\def\im{\mrm{im}}
\def\Tr{\mrm{Tr}}
\def\Z{\mbb{Z}}
\def\R{\mbb{R}}
\def\C{\mbb{C}}
\def\half{\frac{1}{2}}
\def\filler{\phantom{fillerfillerfiller}}
\newcommand{\be}{\begin{equation}}
\newcommand{\ee}{\end{equation}}
\newcommand{\mbb}[1]{\mathbb{#1}}
\newcommand{\mrm}[1]{\mathrm{#1}}
\newcommand{\mcal}[1]{\mathcal{#1}}
\newcommand{\mbf}[1]{\mathbf{#1}}
\newcommand{\ph}[1]{\phantom{#1}}
\newcommand{\udten}[3]{#1^{#2}_{\ph{#2}#3}}
\newcommand{\duten}[3]{#1^{\ph{#2}#3}_{#2}}
%\newcommand{\pd}[2]{\frac{\p#1}{\p#2}}
\newcommand{\D}[2]{\frac{d#1}{d#2}}

\newcommand{\todo}[1]{\color{red}#1\color{black}}

%---Set Margins ------------------------------------------------------
\setlength\oddsidemargin{0.25 in} \setlength\evensidemargin{0.25 in} \setlength\textwidth{6.25 in} \setlength\textheight{8.50 in}
\setlength\footskip{0.25 in} \setlength\topmargin{0 in} \setlength\headheight{0.25 in} \setlength\headsep{0.25 in}



%%%---DOCUMENT---%%%%%%%%%%%%%%%%%%%%%%%%%%%%
\begin{document}

%=== Preliminary Pages ============================================
\begin{frontmatter}
	%%%%%%%%%%%%%%%%%%%%%%%%%%%
% TITLE PAGE INFORMATION %
%%%%%%%%%%%%%%%%%%%%%%%%%%%

\title{Improving Reinforcement Learning for Robotics with Control and Dynamical Systems Theory}

\author{Sean Gillen}

%%%%%%%%%%%%%%%%%%%%%%%%%%%%%%%%%%
% DECLARATIONS FOR FRONT MATTER %
%%%%%%%%%%%%%%%%%%%%%%%%%%%%%%%%%%
\report{Dissertation} \degree{Doctor of Philosophy} \degreemonth{February} \degreeyear{2022}
\defensemonth{March} 
\defenseyear{2022}

\chair{Professor Katie Byl}  % this is your advisor
\othermemberA{Professor Joao Hespanha} % This is a member of your committee 
\othermemberB{Professor Linda Petzold} % This is a member of your committee 
\othermemberC{Professor Yon Vissel} % This is a member of your committee (if your department requires 4 members)
\numberofmembers{4} % should match the number of entries above (chair + othermembers)

\field{Electrical and Computer Engineering}
\campus{Santa Barbara}


%\title{{ University of California \\ Santa Barbara} \linebreak \\  Ph.D. Dissertation}
%\author{Tom\'as Andrade}
%\date{2012}

	\maketitle
	\approvalpage
	\copyrightpage
	\begin{dedication}

\bigskip

${}$ \\

\bigskip

${}$ \\

\bigskip

${}$ \\

\bigskip

\begin{center}
\begin{Large}
Dedicated to future generations seeking to further our collective knowledge, keep pushing. 
\end{Large}
\end{center}


\end{dedication} %comment out if you don't want a dedication
	\begin{acknowledgements}

First and foremost I would like to acknowledge my advisor, Katie Byl, for her support and mentorship throughout my PhD. She welcomed me into her lab, offered me the guidance to get started, and the freedom to explore the many different research topics that I was interested in. I would also like to extend my thanks to the rest of my committee, Joao Hespanha, Linda Petzold, and Yon Visell, for their feedback and support.

I would also like to thank the other members of the UCSB robotics lab that I had the privilege of working beside. Guillaume Bellegarda, Nihar Talele, Asutay Ozmen, Thomas Ibbetson, Roman Aguillera, Sean Anderson, and Chris Cheney. Guillaume and Nihar in particular were the senior members of the lab when I first joined, and they were role models and offered invaluable guidance. Additionally, I would like to acknowledge the visiting students that I had to privilege to work with and mentor, Marco Molnar, Nina Grabka, and Min Dai. 

I also need to thank all of my friends that supported me throughout this time. Both my old friends: Fraser Hood, Haris Godil, Colin Briber, Holden Smith, and Stephanie Dang; and the many new friends I have met during my time in this PhD program: Nathan Tucker, Rahul Chandan, Chris Salls, Chani Jindal, Bryce Ferguson.

%\todo {Decide who else makes the cut ... or if I should make an even shorter list Liz? My Entire DnD group? Keith? Michael Choquer Sharad Shankar , Brian Canty } 

My Mother, Father, and brother Kyle all deserve extensive thanks for their unending support and care. Lastly I thank my partner, Jennifer Chan, for her love, support, and encouragement during final stage of my PhD.


\end{acknowledgements} 
	%\begin{vitae}
\addcontentsline{toc}{chapter}{Curriculum Vitae}

\begin{vitaesection}{Education}
\vspace{-0.1cm}
\item [2021]	Ph.D. in Electircal Engineering (Expected), University of California, Santa Barbara.
\item [2019]	M.A. in Electrical Engineering, University of California, Santa Barbara.
\item [20XX]	etc
\end{vitaesection}

\textbf{Publications}

Publications.

\end{vitae}
    \includepdf[pages=-]{sgillen_CV.pdf}
	%
%  Abstract
%

\begin{abstract}
\addcontentsline{toc}{chapter}{Abstract}
%todo: max 350 words

Deep learning is a very powerful tool for solving all sorts of problems. It however also has many severe drawbacks that hinder it's adoption to the real world. I use knowledge won from other fields to try to attack some of these shortcomings. At times it even appears to be working (if you squint ... generously). I used deep learning to learn to switch between a model based linear controller and a learned controller, and found an application of differentialble physics simulators to the same problem. I then wrote three paper on using post processing to improve reinforcement learning processes. I think the last one was simpler but more important. I wrote a lot of code too, and shared some of it ...

(I also took 2 years of classes, earned a masters degree
and taught for probably another two years combined, including lots of high level classes that I had never taken.
Plus mild server admin stuff
Plus lots of failed ML projects (but now have some ML creds)
Did a freelance software, made some money
Played lots of video games
Wrote lots and lots of code learned several programming languages
Published 5 papers in top conferences
Participation trophy in open source software 
I ended one long term relationship and started another (good decision)
Made many new friends
Got into backpacking (Iceland, peru, georgia), and hiked all over SB
Survived a pandemic
Humbled Myself
)

%\abstractsignature
\end{abstract}



	\tableofcontents
\end{frontmatter}

\begin{mainmatter}

%---Set Headers and Footers ------------------------------------------------------
\pagestyle{fancy}
\renewcommand{\chaptermark}[1]{\markboth{{\sf #1 \hspace*{\fill} Chapter~\thechapter}}{} }
\renewcommand{\sectionmark}[1]{\markright{ {\sf Section~\thesection \hspace*{\fill} #1 }}}
\fancyhf{}

\makeatletter \if@twoside \fancyhead[LO]{\small \rightmark} \fancyhead[RE]{\small\leftmark} \else \fancyhead[LO]{\small\leftmark}
\fancyhead[RE]{\small\rightmark} \fi

\def\cleardoublepage{\clearpage\if@openright \ifodd\c@page\else
  \hbox{}
  \vspace*{\fill}
  \begin{center}
    This page intentionally left blank
  \end{center}
  \vspace{\fill}
  \thispagestyle{plain}
  \newpage
  \fi \fi}
\makeatother
\fancyfoot[c]{\textrm{\textup{\thepage}}} % page number
\fancyfoot[C]{\thepage}
\renewcommand{\headrulewidth}{0.4pt}

\fancypagestyle{plain} { \fancyhf{} \fancyfoot[C]{\thepage}
\renewcommand{\headrulewidth}{0pt}
\renewcommand{\footrulewidth}{0pt}}

%=== Introduction ============================================
\chapter{Introduction and Literature Review}

Reinforcement learning has a long history, which I will tell you about at length.

Now in recent years research into deep reinforcement has exploded, with many impressive results advancing the state of the art in many fields.


- Super human video game AI (starcraft, dota, atari) \\ 
    - Board Games (Go, Chess) (State of the art here ... \todo{Is it} \\ 
    - Object Manipulation (Rubics cube) \\ 
    
    
However there are a number of glaring issues I see with the current state of the art in reinforcement learning.
    - Sample Inefficient \\
    - Sim 2 real must be solved, for robotics \\ 
    - Can't trust \\ 
    - Quantifying Performance is Hard \\ 
    - It is extremely Brittle  \\ 
    - Sensitive To Initial Conditions \\ 
    - Does Terribly in Some Kinds of Problems
    
    
I take some steps to solve this

    I solved the Acrobot, Twice ... Very Cool
    
    I did some reward post processing on a Markov chain thing
    Uhh anyway ... 
    
    
    Only a few years later, we saw a big breakthrough, a team was able to train neural networks to play Atari games at a superhuman level. The input was the raw pixel values from the game, and the reward is the score from the game. This is remarkable for a number of reasons, but it's cool that the same algorithm was able to work successfully across a wide variety of games, which you might think require very different strategies. \cite{mnih2015humanlevel}


% \begin{figure}[!htb]
%     \centering
%     \includegraphics[width=.2\linewidth]{fig/dissertation/Atari.png}
%     \label{fig:atari}
% \end{figure}

\begin{figure}[!htb]
    \centering
    \includegraphics[width=\linewidth]{fig/dissertation/Go.png}
    \caption{\cite{2017Natur.550..354S}}
    \label{fig:alphago}
\end{figure}


\begin{figure}[!htb]
    \centering
    \includegraphics[width=.6\linewidth]{fig/dissertation/humanoid_running.png}
    \caption{\cite{heess_emergence_2017}}
    \label{fig:deepmind_running}
\end{figure}


\begin{figure}[!htb]
    \centering
    \includegraphics[width=.6\linewidth]{fig/dissertation/rubics_cube.png}
    \caption{\cite{openai_learning_2018}}
    \label{fig:rubics_cube}
\end{figure}

\begin{figure}[!htb]
    \centering
    \includegraphics[width=.6\linewidth]{fig/dissertation/anymal_grid.png}
    \caption{\cite{anymal2022}}
    \label{fig:anymal_grid}
\end{figure}


\cite{Carpanese2022} Nucluear fusion


Probably the most impressive and famous examples have been in the realm of games. An early breakthrough was in 2015, when Mnih1 et. al. showed that DRL could play a wide range of Atari video games using deep reinforcement learning \cite{mnih2015humanlevel}. This is significant for two reasons, this first is that this is a very challenging problem, the agent is tasked to play the game using only pixles as input, and only the score of the game as output. Imagine you are the agent, you are handed an array of 12288 numbers (a 64x64 RGB image), asked to pick an action, and then given a score of zero. This repeats maybe 100 times and then you get a reward of one. Was is the action you picked at timestep 62? was it because entry 10456 had a large value at timestep 3? Figuring this out on it's own is impressive, and using the same algorithm to solve dozens of different games was even more impressive. In 2016 Alpha Go achieved super human performance in the game of Go, which no other approach has been able to do \cite{2017Natur.550..354S}. The same team later released Mu Zero, which surpassed Alpha Go and can also play shogi, chess, and a suite of Atari games, again at superhuman levels \cite{muzero2020}. 

in which DRL has been used for physics-based character animation~\cite{2018-TOG-deepMimic}, and a wide variety of complex robotic tasks. This paper focuses primarily on applications in robotics. Continuous control problems in the context of robotics include controlling a 47 degree-of-freedom (DOF) humanoid to navigate various obstacles~\cite{heess_emergence_2017}, dexterously manipulating objects with a 24 DOF robotic hand~\cite{openai_learning_2018}, training the quadrupedal ANYmal robot to recover from falls~\cite{lee_robust_2019} \cite{hwangbo_learning_2019}, and teaching the bipedal Cassie robot to navigate stairs blindly~\cite{siekmann2021blind}. Recently I might even argue that DRL has taken over Boston dynamics in terms of robust walking with the recent work by ETH Zurich \cite{anymal2022}.


    


%=== Chapter  ============================================
\chapter{Background}
%---  Section -------------------------

\section{Reinforcement Learning}
\begin{figure}[!htb]
    \centering
    \includegraphics[width=\linewidth]{fig/dissertation/RL_sutton.png}
    \caption{The classic picture from Sutton and Barto that everyone seems to steal without attribution}
    \label{fig:r}
\end{figure}

What is reinforcement learning? Let's introduce the basic terminology first, we will formalize this later. An agent takes actions in an environment in order to maximize a reward. The environment is described by a state, which can be observed (sometimes only partially) by the agent. One example might be a chess playing program, the state is the position of the pieces on the board, the action the their move on the current term, and their reward is a one when they win the game and a zero otherwise. Or maybe the agent is a robot, whose states are joint angles and velocities, whos actions are motor commands, and whos reward function is their forward velocity. Another example might be a motor inside that very robot, it's state might include a motor command and a measurement of the current motor position, and it's output might be a voltage signal.

As these examples hopefully illustrate, reinforcement learning is extremely general, and can and has been applied to a myriad of problems across a number of domains. 

A history of reinforcement learning... Do I need this? 
I can just present the state of the art here to compare to what I did. And I can do that in chronological order

Definition and Problem Statement, maybe some early victories
\todo{Also need to indtroduce Q function and Value functions}

In reinforcement learning, the goal is to train an agent, acting in an environment, to maximize some reward function. The environment is a discrete time dynamical system described by state $s_{t} \in \mathbb{R}^{n}$ and the current action $a_{t} \in \mathbb{R}^{b}$. an evolution function $f: \mathbb{R}^{n} \times \mathbb{R}^{b} \rightarrow \mathbb{R}^{n}$ takes as input the current state and action, and outputs the state at time t+1:


\begin{equation}
s_{t+1}= f(s_{t},a_{t})
\end{equation} 

The controller is the function we are learning $g: \mathbb{R}^{n} \times \mathbb{R}^{\norm{\theta}} \rightarrow \mathbb{R}^{m}$ such that:

\begin{equation}
a_{t} = g(s_{t}, \theta)
\end{equation} 

The goal is to maximize a reward function $r : \mathbb{R}^{n} \times \mathbb{R}^{m} \times \mathbb{R}^{n} \rightarrow \mathbb{R}$. We consider the finite time undiscounted reward case. The objective function then is:

\begin{equation} 
R(\theta) =  \sum_{t=0}^{T}r(s_{t}, a_{t}, s_{t+1}) 
\end{equation}


Then REINFORCE

REINFORCE is the OG policy gradient algorithm, it's pretty old and influential. \todo{I need to introduce policy gradients here I guess}

Then some stuff I don't care about

\section{Parameterized Functions and Neural Networks}

It's worth taking a second here and talking about so called function approximators and neural networks, because otherwise the rest of this paper won't make any sense. A function approximator is really just a function which has tunable parameter in addition to inputs and outputs. In controls or reinforcement learning, we usually just call these "parameterized policies". Let's look at a simple example, third order polynomial with coefficients $a_{0}$, $a_{1}$, and $a_{2}$

\begin{equation}
\label{eq:poly}
    y = a_{0} + a_{1}x + a_{2}x^{2}
\end{equation}

At risk of belaboring the point, the equation above has an input (x), and output(y) but also parameters($a$). In function approximation, we typically have data from an unknown function, often sampled from a physical system, and we wish to solve for parameters such that our function approximator best matches the output of the target function. 

One of the most successful and popular function approximators, especially in recent years, have been neural networks. These come in many flavors, multilayers perecptrons (MLPs), convolutional neural networks (CNNs), long short term memories (LSTM) and Gated Recurrent Units (GRUs).

Let's examine the MLP first in particular, which is the most common especially for control applications. We can see a diagram for an MLP in figure \ref{fig:mlp}

\begin{figure}[!htb]
    \centering
    \includegraphics[width=.8\linewidth]{fig/dissertation/mlp.png}
    \caption{A very simple MLP}
    \label{fig:mlp}
\end{figure}

The outut is computed as follows: \todo{blah blah}

So essentially, we do a big matrix multiply of the weights, a vectorized nonlinearity on the result, a new matrix multiply with the next layer weights, repeat. 

Cool, another thing worth noting is that for most modern DRL, the thing we are parameterizing is actually a probability over actions rather than directly mapping to actions. So if we have an action space of size three then our network may have 6 outputs, 3 means and 3 standard deviations for normal distributions over the actions. When it is time to select an action, we sample from this distribution. 

\section{Deep Learning and Supervised Learning}

Supervised learning is when you learn by example. For example classifying images as either dogs or cats. This is done with backpropogation and a certain loss

\todo{put in cross entropy loss from CDC paper}

As discussed in the introduction, deep supervised learning really took off starting around 2012. 


\section{Deep Reinforcement Learning}
Sometimes around 2012 or so, deep neural networks saw a resurgence of interest in the space of supervised learning. This is usually attributed to the success of the work of Alex Krizhevsky who showed that convolutional neural networks were extremely effective at image classification \cite{NIPS2012_c399862d}. 




There are sort of two schools of thought when it comes to RL for locomotion. There's RL for the animation of physics based characters, which is mostly useful in the context of movies or video games. This is mostly centered in michael v. panns group (\todo{probably}) at 
Then there is everyone else, who is in it for robotics, or just for pure RL maybe ... 

Google brain, deepmind, openAI, Berkely AI lab, these are the most prolific labs in this space. 

Still need to introduce openAI gym, and dm control suite...
And also talk about the sort of two schools of thought, character animation and robotic control. 

Here's where we get to the part that introduced me to the topic. In 2017 Deep Mind released some work regarding locomotion policies in simulated environments, they look a little silly, but I thought it was mind blowing at the time, since I was working with a simpler system, and couldn't get it to work at. all. 

Then Deep Mind Mujoco, PPO, Lots of stuff
    
    
We can further divide up DRL into on policy and off policy algorithms. On policy algorithms use only data obtained with the under the current policy to make policy updates. After an update it essentially throws out the data obtained so far and starts afresh. Examples of this include the classic REINFORCE, A2C, TPRO and PPO. Off policy algorithms by contrast use data obtained by previous policies to continue updating. This usually takes the form of a "replay buffer" which stores state, action, and reward tuples obtained so far during training. These tuples can then be used to update the current policy. Off policy algorithms are often a variant of Q learning, since the Q function in reinforcement learning  does not require the current action, it is well suited to off policy learning. 

In general, off policy algorithms tend to be more sample efficient, requiring less samples to obtain a similar reward level. On policy algorithms are less sample efficient, but do tend to be more stable, getting more consistent rewards across random seeds.

A Word on Model Based Reinforcement Learning:
    - Not covered here, but here's what it would be 



%=== Chapter ==============================================
\chapter{Switching}

%%%%%%%%%%%%%%%%%%%%%%%%%%%%%%%%%%%%%%%%%%%%%%%%%%%%%%%%%%%%%%%%%%%%%%%%%%%%%%%%
\section{Introduction}

Advances in machine learning have allowed researchers to leverage the massive amount of compute available today in order to better control robotic systems. The result is that modern model-free reinforcement learning has been used to solve very difficult problems. Recent examples include controlling a 47 DOF humanoid to navigate a variety of obstacles \cite{heess_emergence_2017}, dexterously manipulating objects with a 24 DOF robotic hand \cite{openai_learning_2018}, and allowing a physical quadruped robot to run \cite{hwangbo_learning_2019}, and recover from falls \cite{lee_robust_2019}.

Despite this, these algorithms can struggle on certain low dimensional problems from the nonlinear control literature. Namely  the acrobot \cite{spong_swing_1994} and the cart pole pendulum. These are both under-actuated mechanical systems that have unstable fixed points in their unforced dynamics (see section \ref{section:Acrobot}). Typically, the goal is to bring the system to this fixed point and keep it there. In this paper we focus on the acrobot as we found less examples of model free reinforcement learning performing well on this task.

\begin{figure}[ht]
\centering
  \includegraphics[scale=.25]{fig/cdc2020/system5.png}
  \caption{System diagram for the new technique proposed in this paper. Rounded boxes represent learned neural networks, squared boxes represent static, hand crafted functions. The local controller is a hand designed LQR, the swing-up controller is obtained via reinforcement learning, and the gating function is trained as a neural network classifier}
  \label{fig:hyst}
\end{figure}


It is not uncommon to see some variation of these systems tackled in various reinforcement benchmarks, but we have found these problems have usually been artificially modified to make them easier. For example the very popular OpenAI Gym benchmarks \cite{1606.01540} includes an acrobot task. But the objective is only to get the system in the rough area of the unstable fixed point, and the dynamics are integrated with a fixed time-step of .2 seconds, which makes the problem much easier and unrepresentative of a physical system. We have found that almost universally, modern model free reinforcement learning algorithms fail to solve a more realistic version of the task. Notably, the Deep Mind control suite \cite{deepmindcontrolsuite2018} includes the full acrobot problem, and all but one algorithm that they tested (the exception being \cite{barth-maron_distributed_2018}) learned nothing, the average return after training was the same as before training. 

Despite this, there are many traditional model based solutions \cite{spong_swing_1994}, \cite{spong_energy_1996}, that can solve this problem well. In this work we do not seek to improve upon the model based solutions to this problem, but to extend to the class of problems that model free reinforcement learning methods can be used to solve. We believe the methods used here to solve the acrobot can be extended to other problems, such as making robust walking policies.

 One of the primary reasons why this problem is difficult for RL is that the region of state space that can be brought to the unstable fixed point is very small, even with generous torque limits. An untrained RL agent explores by taking random actions in the environment. Reaching the region of attraction is rare, we found that for our system, random actions will reach the basin of attraction for a well designed LQR in about 1\% of trials. However an RL agent doesn't have access to a well designed LQR at the start of training, in addition to reaching the region where stabilization is possible, the agent must  also stabilize the acrobot for the agent to receive a strong reward signal. This results in successful trials in this environment being extremely rare, and therefore training is in-feasibly slow and sample inefficient.

Our solution to add a predesigned balancing controller into the system, this is comparatively easy to design, and can be done with a linear controller. Our contribution is a novel way to combine this balancing controller with an algorithm that is learning the swing-up behavior. We simultaneously learn the swing-up controller, and a function that switches between the two controllers.  

\subsection{Related Work}

Work done by Randolov et. al. \cite{randlov_combining_2000} is closely related to our own. In that work they construct a local controller, an LQR, and combine it with a learned controller to swing-up and balance an inverted double pendulum (similar to the acrobot we study but with actuators at both joints). The primary differences between our work and theirs is that they hard code the transition between their two controllers. In contrast we learn our transition function online and in parallel with our swing-up controller.


% Also, the learning algorithm they use, SARSA($\lambda$), is limited to discrete action spaces, and therefore to apply this technique to continuous action spaces requires an explicit discretization of the action space beforehand, which both limits the flexibility of the controller and increases the design effort.  

Work done by Yoshimoto et. al. \cite{yoshimoto_acrobot_2005}, like ours, learns the transition function between controllers in order to swing-up and balance an acrobot. However, unlike our work they limit the controllers they switch between to pre-computed linear functions. In contrast our work simultaneously learns a nonlinear swing-up controller and the transition between a learned and pre-computed balance controller.

%Like before, restricting the control to pre-computed controllers limits the applicability of this method to more complex systems, where an optimal, or even good enough local controller may not be known ahead of time. 

Wiklednt et. al  \cite{wiklendt_small_2009} too swing-up and balance an acrobot using a combined neural network and LQR. However they only learn to swing-up from a single initial condition, whereas our method learns to solve the task from any initial position.


%% Can include this, need too? !!!
%More recently, Lee et. al. \cite{lee_robust_2019} has done work in selecting between discrete, learned, behaviors using reinforcement learning. They apply this to teach a physical quadruped to recover from a fall, with very impressive results. Their approach does differ from ours though, in that each behavior is learned independently and then the transitions between them learned separately. In our case it turned out to be extremely important to train the swingup controller while the balance controller was on and being switched to. Without this the swingup policy tends to not learn anything.

Doya \cite{doya_multiple_2002} also learns many controllers using reinforcement learning, and adaptively switches between them. However unlike our work, the switching function is not learned using reinforcement learning, but is instead selected according to which of the controllers currently makes best prediction of the state at the current point in state space. We believe our model free updates will avoid the model bias that can be associated with such approaches. Furthermore our work allows for combining learned controllers with hand designed controllers, such as the LQR.  

\section{The Acrobot System}

\label{section:Acrobot}

The acrobot is described in Figure \ref{fig:acrobot}. It is a double inverted pendulum with a motor only at the elbow. We use the parameters from Spong \cite{spong_swing_1994}:

\begin{center}
%\captionof{table}{Mass and inertial parameters used in simulation}
\begin{tabular}{ | c | c | c | }
\hline
Parameter & Value & Units\\
 \hline
 $m_{1}, m_{2}$ & 1 & Kg \\ 
 \hline
 $l_{1}, l_{2}$ & 1 & m  \\ 
 \hline
 $l_{c1}, l_{c2}$  & .5 & m  \\ 
 \hline
 $I_{1}$ & .2 & Kg*m$^{2}$ \\
 \hline 
 $I_{2}$ & 1.0 & Kg*m$^{2}$  \\ 
 \hline
\end{tabular}
\end{center}

\begin{figure}[ht]
\centering
  \includegraphics[scale=.25]{fig/cdc2020/acrobot2.png}
  \caption{Diagram for the acrobot system}
  \label{fig:acrobot}
\end{figure}


The state of this system is $s_{t} = [\theta_{1}, \theta_{2}, \dot \theta_{1}, \dot \theta_{2}]$. The action $a_{t} = \tau$, is the torque at the elbow joint. The goal we wish to achieve is to take this system from any random initial state, to the upright state $gs = [\pi/2, 0, 0, 0]$, which we will refer to as the goal state. To achieve this goal, we seek the maximum of the following reward function:

\begin{dmath} r_{t} = l_{1}\sin(\theta_{1}) + l_{2} \sin(\theta_{1} + \theta_{2})
\end{dmath}
This was motivated by the popular Acrobot-v1 environment \cite{baselines}, We found empirically that for our algorithm this reward signal led to the same solutions as the more typical $\norm{s_{t} - gs}$. However we found that some of the other algorithms we compared to perform better with the sinusoidal reward function. 

We implement the system in python (all source code is provided, see footnote on page one), the dynamics are implemented using Euler integration with a time-step of .01 seconds, and the control is updated every .2 seconds. We experimented with smaller timesteps and higher order integrators, generally we found these made the balancing task easier, but made the wall clock time for the learning much slower.  



% policy = $\pi_{\theta}$ \\
% gate   = $G_{\gamma}$ \\
% value = $V_{\phi}$ \\
% target value = ${V}_{\overline{\phi}}$ \\
% Q1 = $Q_{\rho_{1}}$ \\
% Q2 = $Q_{\rho_{2}}$ \\



\section{Switched Soft Actor Critic}
\label{section:SSAC}

Our primary contribution is to extend SAC in two key ways, we call the modified algorithm switched soft actor critic (SSAC). The first modification is a change to the structure of the learned controller in order to inject our domain knowledge into the learning. Our controller consists of three distinct components. The gate function, the balancing controller, and the swing-up controller. The gate, $G_{\gamma}: S \rightarrow [0,1]$, is a neural network parameterized by weights $\gamma$ which takes the observations at each time step and outputs a number $g_{t}$ representing which controller it thinks should be active. $g_{t} \approx 1$ implies high confidence that the balancing controller should be active, and $g_{t} \approx 0$ implies the swing-up controller is active. This output is fed through a standard switching hysteris function, to avoid rapidly switching on the class boundary, parameters given in the appendix. The swing-up controller can be seen as the policy network from vanilla SAC, the action then is determined by equation (\ref{act}). The parameters for these networks are given in the appendix. The balancing controller is a linear quadratic regulator $C: S \rightarrow A$  about the acrobot's unstable equilibrium. We use the LQR designed by Spong \cite{spong_swing_1994}:

Using 

\[ Q = \begin{pmatrix} 1000 & -500 & 0 & 0 \\ -500 & 1000 & 0 & 0 \\ 0 & 0 & 1000 & -500 \\ 0 & 0 & -500 & 1000\end{pmatrix}, R = \begin{pmatrix} .5 \end{pmatrix}
\]

The resulting control law is: 
\[ u = -Ks \]

with 

\[ K = [-1649.8,  -460.2,  -716.1,  -278.2] \] 

These three functions together form our policy,  $\pi_{\theta}$. Algorithm \ref{alg:rollout} demonstrates how the action is computed at each timestep.

We learn the basin of attraction for the regulator by framing it as a classification problem, our neural network takes as input the current state, and outputs a class prediction between 0-1. A one implying that the LQR is able to stabilize the system, and a zero implying that it cannot. We then define a threshold function $T(s)$, as a criteria for what we consider a successful trial:

\begin{dmath} {T(s) = \norm{s_{t} - gs } < \epsilon_{thr}} \quad \forall t \in \{N_{e}-b, ...,  N_{e} \}  \label{T} 
\end{dmath}

Here $s$ is understood to be an entire trajectory of states, $N_{e}$ is the length of each episode, $e_{thr}$ and $b$ hyper parameters with values given in the appendix. We are following the convention of a programming language here, (\ref{T}) returns one when the inequality holds, and zero otherwise. To gather data, we sample a random initial condition, do a policy roll out using the LQR, and record the value of \ref{T} as the class label.

To train the gating network we minimize the binary cross entropy loss:

\begin{dmath}L^{\text{G}} =  \mathop{\mathbb{E}}_{\gamma} -\left[ c_{w} y_{i}\log(G_{\gamma}(s_{i})) + (1 - y_{i})\log(1 - G_{\gamma}(s_{i})) \right]\end{dmath} 

Where $y_{i}$ is the class label for the ith sample, $c_{w}$ is a class weight for positive examples. we set $c_{w} = \frac{n_{t}}{n_{p}}w $ where $n_{t}$ is the total number of samples, $n_{p}$ is the number of positive examples, and $w$ is a manually chosen weighting parameter to encourage learning a conservative basin of attraction. We found that the learned basin was very sensitive to this parameter, a value of .01 empirically works well. Note that unlike the other losses above, the data here is not computed over a sample but is instead computed over the entire replay buffer. We found the gate was prone to "forgetting" the basin of attraction early in the training otherwise. This also allows us to update the gate infrequently compared to the other networks, and so the total impact on wall clock time is modest.

The second extension is a modification of the replay buffer $D$. We do this by constructing $D$ from two separate buffers, $D_{n}$ and $D_{r}$. Only roll outs that ended in a successful balance (as defined by equation (\ref{T})) are stored in $D_{r}$. The other buffer stores all trials, the same as the unmodified replay buffer. Whenever we draw experience from $D$, with probability $p_{d}$ we sample from $D_{n}$, and with probability $(1-p_{d})$ we sample from $D_{r}$. We found this to speed up learning dramatically, as even with the LQR and a decent gating function in place, the swing-up controller finds the basin of attraction only in a tiny minority of trials.


\begin{algorithm}
\caption{Warm Start Gate Data Generation}\label{euclid}
\begin{algorithmic}[1]
\State Initialize network weights $\gamma$ 
\For {$i \in \{1, ..., N_{d}\} $}
\State  sample initial state $s_{0}$ from observation space
\If {$s_{i} \in B$}
\State    $y_{i} = 0$
\Else
\State    $y_{i} = 1$
\EndIf
\EndFor
\State set $\alpha(0) = 1$, and $\alpha(1) = \frac{\text{sum(y)}}{\text{len(y}}$
\State update $G_{\gamma}$ with $n_{w}$ steps of Adam to minimize $L^{ws}$
\end{algorithmic}
\end{algorithm}

% policy = $\pi_{\theta}$ \\
% gate   = $G_{\gamma}$ \\
% value = $V_{\phi}$ \\
% target value = ${V}_{\overline{\phi}}$ \\
% Q1 = $Q_{\rho_{1}}$ \\
% Q2 = $Q_{\rho_{2}}$ \\


\begin{algorithm}
\caption{Do-Rollout($G_{\gamma}, \Pi_{\theta}$, K)}
\label{alg:rollout}
\begin{algorithmic}[1]
\State $s = r = a = g = r = \{\}$
\State  Reset environment, collect $s_{0}$
\For    {$t \in \{0, ..., T\} $}
\State  $g_{t} = hyst(G_{\gamma}(s_{t}))$ 
\If {$(g_{t}) == 1$}
\State    $a_{t} = -Ks_{t}$
\Else
\State   Sample $\epsilon_{t}$ from $N(0, 1)$
\State   $a_{t} = \beta\tanh(\mu_{\theta}(s_{t}) + \sigma_{\theta}(s_{t})*\epsilon_{t})$ 
\EndIf
\State   Take one step using $a_{t}$,  collect $\{s_{t+1}, r_{t}\}$
\State   $s = s \bigcup s_{t}$, $r = r \bigcup r_{t}$
\State   $a = a \bigcup a_{t}$,  $g = g \bigcup g_{t}$
\EndFor
\State \bf{return} $s, a, r, g$
\end{algorithmic}
\end{algorithm}


\begin{algorithm}
\label{algo:SSAC}
\caption{Switched Soft Actor Critic}\label{euclid}
\begin{algorithmic}[1]
\State Initialize network weights $\theta ,\phi, \gamma, \rho_{1}, \rho_{2}$ randomly
\State set $\overline \phi = \phi$
\For{$n \in \{0, ..., N_{e}\} $}
\State $s,r,a,g = \text{Do-Rollout}(G_{\gamma}, \Pi_{\theta}, K)$
\If {$T(s)$}
\State Store $s,r,a$ in $D_{n}$
\EndIf
\State Store $s,r,a$ in $D_{r}$
\State Store $s,g,T(s)$ in $D_{g}$
\If {Time to update policy}
\State sample $s^{r}, a^{r}, r^{r}$ from $D$
\State $\hat Q \approx R + \gamma V_{\overline{\phi}}(S)$
\State $Q^{min} = \min(Q_{\rho_{1}}(s^{r},a^{r}), Q_{\rho_{2}}(s^{r},a^{r}))$
\State $\hat V \approx Q^{min} - \alpha H(\pi_{\theta} (A|S))$
\State Run one step of Adam on $L^{Q}(s^{r}, q^{r}, r^{r})$
\State Run one step of Adam on $L^{\pi}(s^{r})$
\State Run one step of Adam on $L^{V}(s^{r})$
\State   $\overline \phi =  q \overline \phi + (1-q)\phi$
\EndIf
\If {Time to update gate}
\State Run one step of Adam on $L^{G}$ using all samples in $D_{g}$
\EndIf
\EndFor
\end{algorithmic}
\end{algorithm}

\section{Results}

\subsection{Training} 
To train SSAC we first start by training the gate exclusively, using the supervised learning procedure outlined in section \ref{section:SSAC} This allows us to form an estimate of the basin of attraction before we try to learn to reach it. We trained the gate for 1e6 timesteps, and then trained both in parallel using algorithm 2 for another 1e6 timesteps. The policy, value, and Q functions are updated every 10 episodes, and the gate every 1000. The disparity is because, as mentioned earlier, the gate is updated using the entire replay buffer, while all the other losses are updated with one sample batch from the buffer. Hyperparameters were selected by picking the best performing values from a manual search, which are reported in the appendix.

In addition to training on our own version of SAC and Switched SAC we also examined the performance of several algorithms written by OpenAI and cleaned up by the community \cite{stable-baselines}. We examine PPO and TRPO, two popular trust region methods. A2C was included to compare to a non trust region, modern policy gradient algorithm. We also include TD3, which has been shown in the literature to do well on the acrobot and cartpole problems \cite{lillicrap_continuous_2015}

Stable baselines includes hyperparameters that were algorithmically tuned for each environment. For algorithms where parameters for Acrobot-v1  were available we chose those, some algorithms were missing tuned Acrobot-v1 examples, and for those we used parameters for Pendulum-v0, simply because it is another continuous, low dimensional task. Note we don't expect the hyper-parameters to impact the learned policy's score in this case, only how fast learning occurs. Reported rewards are averaged over 4 random seeds. Every algorithm makes 2e6 interactions with the environment. Also note that this project was largely inspired by spending a large amount of time manually tuning these parameters to work on this task (with no success better than what we see here). Figure \ref{fig:switched_reward} shows the reward curve for our algorithm and the algorithms from stable baselines. Table \ref{table:results} shows the mean and standard deviation for the final rewards obtained by all algorithms. 

 \begin{figure}[h]
\centering
  \includegraphics[scale=.5]{fig/cdc2020/reward.png}
  \caption{Reward curve for SSAC and the other algorithms we compare to. the solid line is the smoothed average of episode reward, averaged over four random seeds. The shaded area indicates the best and worst rewards at each epoch across the four seeds. SSAC is shown starting later to account for the time training the gating function alone.}
  \label{fig:switched_reward}
\end{figure}


\begin{center}
% If you know how to tell this environment to not pagebreak that would be great
\begin{tabu}{| X[l] | X[l] |}
\hline
Algorithm (implementation) & Mean Reward $\pm$ Standard Deviation\\
 \hline
 SSAC (Ours) & \bf{92.12 $\pm$ 2.35}  \\
 \hline
 SAC & 73.01 $\pm$  11.41 \\
 \hline
 PPO  &  0.43 $\pm$ 8.89 \\ 
 \hline
 TD3  & 78.67  $\pm$ 61.85 \\
 \hline
 TRPO  &  17.63 $\pm$ 3.39 \\
 \hline
 A2C  & 2.57 $\pm$ 3.63 \\
 \hline
\end{tabu}
%\caption{table}{Rewards after training for across learning algorithms. This table shows results after 2 million environment interactions}
\label{table:results}
\end{center}

As we can see, for this environment, with the number of steps we have allotted, our approach outperforms the algorithms we compared to, with TD3 making it the closest to our performance. This is a necessarily flawed comparison. These algorithms are meant to be general purpose, so it is unfair to compare them to something designed for a particular problem. But that is part of the point we are making, that adding just a small amount of domain knowledge can improve performance dramatically.

\subsection{Analyzing performance}

To qualitatively evaluate the performance of our learned agent we examine the behavior during individual episodes. SSAC gives us a deterministic controller (we can set $\epsilon_{t}$ from \ref{act} to zero). We chose the initial condition $s_{0} = (-\pi/2, 0, 0, 0)$ and record a rollout. The actions are displayed in figure \ref{fig:act}, and the positions in \ref{fig:obs}.

 \begin{figure}[h!]
\centering
  \includegraphics[scale=.5]{fig/cdc2020/act_hist.png}
  \caption{Torque exerted during the sampled episode}
  \label{fig:act}
\end{figure}

\begin{figure}[h!]
\centering
  \includegraphics[scale=.5]{fig/cdc2020/obs_hist.png}
  \caption{Observations during the sampled episode}
  \label{fig:obs}
\end{figure}

We have also found that despite achieving relatively high rewards, the other algorithms we compare to often fail to meet the balance criteria (11). We often see solutions where the first link is constantly rotating, with the second link constantly vertical. To demonstrate this, as well as to demonstrate our algorithms robustness, we run roll outs with the trained agents across a grid of initial conditions, recording if the trajectory satisfies (11) or not. We compare our method with TD3, which was the best performing model free method we could find on this task. Figure \ref{fig:td3_map} show the results, \textbf{when these initial conditions were run for SSAC, it satisfied (11) for every initial condition}. 

\begin{figure}[h]
\centering
  \includegraphics[scale=.35]{fig/cdc2020/th_map_td3.png}
  \caption{Balance map for TD3, X and Y indicate the initial position for the trial, a black dot indicates that the trial started from that point satisfies equation (\ref{T}), and red indicates the converse. \textbf{when these initial conditions were run for SSAC, it satisfied (11) for every initial condition}}
  \label{fig:td3_map}
\end{figure}



\section{Conclusions}

We have presented a novel control design methodology that allows engineers to leverage their domain knowledge, while also reaping many of the benefits from recent advances in deep reinforcement learning. In our case study we constructed a policy to swing-up and balance an acrobot while only needing to manually design a linear controller for the balancing task. We believe this method of control will be straightforward to apply to the double or triple cartpole problems, which to our knowledge no model free algorithm is reported as solving. We also think that this general methodology can be extended to more complex problems, such as legged locomotion. In that case the linear controller here could be a nominal walking controller obtained via trajectory optimization, and the learned controller could be a recovery controller to return to the basin of attraction of this nominal controller. 

\section*{APPENDIX}

\subsection*{Hyperparameters}

\begin{center}
\begin{tabu}{ X[2,l] | X[1,l] }
Hyperparameter & Value \\
 \hline
 Episode length ($N_{e}$) & 50  \\ 
 Exploration steps & 5e4 \\
 Initial policy/value learning rate & 1e-3  \\ 
 Steps per update & 500 \\
 Replay batch size & 4096 \\
 Policy/value minibatch size & 128 \\ 
 Initial gate learning rate & 1e-5  \\ 
 Win criteria lookback (b) & 10 \\
 Win criteria threshold ($\epsilon_{thr}$) & .1 \\ 
 Discount ($\gamma$)   & .95 \\
 Policy/value updates per epoch & 4 \\
 Gate update frequency & 5e4 \\
 Needle lookup probability $p_{n}$ & .5 \\ 
 Entropy coefficient ($\alpha$) & .05 \\
 Polyak constant ($c_{py}$)  & .995\\
 Hysteresis on threshold & .9 \\
 Hysteresis off threshold & .5 \\
 \hline
\end{tabu}
\end{center}

\subsection*{Network Architecture}
The policy, value, and Q networks are all made of four fully connected layers, with 32 hidden nodes and Relu activations. The gate network is composed of two hidden layers with 32 nodes each, also with Relu activations, the last output is fed through a sigmoid to keep the result between 0-1.




%=== Chapter   ============================================
\chapter{Reward Post Processing and Mesh Dimensions}


   
    The availability of computation as a resource has been growing exponentially since at least the 1970s, and there is every indication that this resource will continue to become cheaper and more available well into the conceivable future. Researchers have been able to leverage the large amounts compute available to better control robotic systems, and advances in computational capacity and algorithmic development continue to open up new domains. One promising manifestation of this is model-free reinforcement learning, a branch of machine learning which allows an agent to interact with its environment and autonomously learn how to maximize some measure of reward. The promise here is to allow researchers to solve problems for systems that are hard to model, and/or that the user doesn't know how to solve themselves.  Recent examples in the context of robotics include controlling a 47 DOF humanoid to navigate a variety of obstacles \cite{heess_emergence_2017}, dexterously manipulating objects with a 24 DOF robotic hand \cite{openai_learning_2018}, and allowing a physical quadruped robot to run \cite{hwangbo_learning_2019}, and recover from falls \cite{lee_robust_2019}.

    In this paper we study legged locomotion. This class of problems is notoriously difficult, and as a result reinforcement learning is a popular tool to throw at it. We would argue that hand-designed, model based control still represents the state of the art (a la Boston Dynamics), but RL has been a fruitful approach. There are examples of learned policies outperforming hand designed ones \cite{hwangbo_learning_2019}, and there is good reason to believe these learning methods will continue their current trajectory of increasing performance gains and ease of use. But these algorithms have a serious draw back in that they are mostly black boxes. It is an open challenge to figure out what exactly it is that your RL agent has learned. If all you know is that one of your agents achieved very high reward, it is not clear how to verify that this system is safe and sensible in all the regions of state space it will visit during its life. Nor can we necessarily say anything about the stability or robustness properties of the system. Recent work ~\cite{Taleledeep} has used so-called mesh-based tools to examine precisely these questions. 
    
    However, utility of any mesh-based tool to accurately discretize a state-space is limited, due to the curse of dimensionality. In practice, these methods are only able to work on relatively high dimensional systems if the reachable space grows at a rate that is much smaller than the exponential growth of the full state space the system within which it is embedded. To expand these methods to higher dimensional systems. We will need to find ways to keep the volume of visited states from expanding commensurately. One way to quantify this rate of growth is by using one of the several notions of "fractional dimensions" from fractal geometry.
    
    In this work, we discuss an efficient meshing algorithm, which we call box meshing. We show that this approach makes calculating the so called mesh dimension feasible in the context of reinforcement learning. We also propose using other notions of fractional dimension from the literature as a proxy for the property we care about. We then show that reinforcement learning agents can be trained to shrink these measures by post processing their reward function. We present the results of this training, and finally present some brief analysis of the resulting structure for select policies.  


    

\section{Meshing \& Fractional Dimensions}


\begin{figure}[h!]
  \centering
  \begin{subfigure}[b]{0.32\textwidth}
    \includegraphics[width=\textwidth]{fig/corl2020/acc_fractal/Fractaldimensionexamplebw.png}
    \caption{Scaling in different dimensions}
  \end{subfigure}
  \begin{subfigure}[b]{0.32\textwidth}
    \includegraphics[width=\textwidth]{fig/corl2020/acc_fractal/Snow_Mesh_Example.png}
    \caption{A non uniform mesh}
  \end{subfigure}
  \begin{subfigure}[b]{0.32\textwidth}
    \includegraphics[width=\textwidth]{fig/corl2020/acc_fractal/Snow_LineFitAnn.png}
    \caption{Calculating the mesh dimension}
  \end{subfigure}
  \caption{Image credit for sub-figure a: \cite{BrendanRyan/Publicdomain2020}, image credit for sub-figures b and c: \cite{Talelepush}}
  \label{fig:fracdim}
\end{figure}

% \begin{wrapfigure}{L}{0.3\textwidth}
% \centering
% \includegraphics[width=0.3\textwidth]{fig/corl2020/HalfCheetahRews.png}
% \caption{\label{fig:frog2} Half Cheetah Reward Curves.}
% \end{wrapfigure}

%(??) is there other work I should be citing here?
%(??) I feel like this section is already too long, but also if I don't explain the pertubation style meshing people will misunderstand what we are tying to do and what "meshing" is?
Let's say we have a continuous set S that we want to approximate by selecting a discrete set M  composed of regions in S. We will call this set M a mesh of our space. Figure \ref{fig:fracdim}(a) shows some examples of this: a line is broken into segments, a square into grid spaces, and so on. The question is: as we increase the resolution of these regions, how many more regions N do we need? Again, figure \ref{fig:fracdim}(a) shows us some very simple examples. For a D dimensional system, if we go from regions of size d to d/k, then we would expect the number of mesh points to scale as $N \propto k^{D}$. But not all systems will scale like this, as \ref{fig:fracdim}(b) and \ref{fig:fracdim}(c) illustrate. Figure \ref{fig:fracdim}(b) is an example of a curve embedded in a two dimensional space. The question of how many mesh points are required must be answered empirically. Going backwards, we can use this relationship to assign a notion of "dimension" to the curve. 

\begin{equation}
    D_{f} = -\lim_{k \rightarrow 0}\frac{\log N(k)}{\log k} 
\end{equation}

What we are talking about is called the Minkowski–Bouligand dimension, also known as the box counting dimension. This dimension need not be an integer, hence the name "fractional dimension". As a practical matter, we use the slope of the log-log plot of mesh sizes over d to calculate this, rather than taking a limit. This is one of many measures of "fractional dimension" that that emerged from the study of fractal geometry. Although these measures were invented to study fractals, they can still be usefully applied to non-fractal sets.

In \cite{OguzSaglam2015}, Saglam and Byl introduced a technique that is able to simultaneously build a non-uniform mesh of a reachable state space while developing robust policies for a bipedal walker on rough terrain. Having a discrete mesh allows for value iteration over several candidate controllers, which found a robust control policy. In addition this mesh allows for the construction of a state transition matrix, which was used to calculate the mean first passage time \cite{Byl2009}, a metric that quantifies the expected number of steps a meta-stable system can take before falling. 

Since its introduction, meshing in this fashion has been used for designing walking controllers robust to push disturbances \cite{Talelepush}, to design agile gaits for a quadruped \cite{Byl2017}, and to analyze hybrid zero dynamics (HZD) controllers \cite{Saglam2015}. There has also been recent work to use these tools to analyze policies trained by deep reinforcement learning \cite{Taleledeep}. A long term goal and motivation for this work is to take a high performance controller obtained via reinforcement learning, and extract from it a mesh-based policy that is both explainable and amenable to analysis.

\subsection{Box Meshing}
\label{sec:boxmsh}
Our primary improvement to the prior work on meshing is to introduce something we call box meshing. Prior, a new mesh point could take any value in the state space. To determine if a new state is already in the mesh, we would compute a distance metric to every point in the mesh, and check if the minimum was below our threshold. Thus, building the mesh was an $O(n)^{2}$ algorithm. By contrast, in box meshing we apriori divide the space uniformly into boxes with side length $d$. We identify any state s with a key obtained by: $\text{key} = \text{round}(\frac{s}{d})d$, where round performs an element-wise rounding to the nearest integer. We can then use these keys to store mesh points in a hash table. Using this data structure, we can still store the mesh compactly, only keeping the points we come across. However, insertion and search are now $O(1)$, and so building the mesh is $O(n)$. This is very similar to non-hierarchical bucket methods, which are well studied spatial data structure \cite{Samet1990}, although we are using them for data compression here. In the prior meshing work, this sort of speedup would be minor, the run-time is dominated by the simulator or robot. However, this speedup does open some new possibilities: most poignantly, it makes calculating the mesh dimension during reinforcement learning plausible. 


\subsection{Algorithmic Box Mesh Dimension}
\label{sec:boxdim}
The "mesh dimension" is the quantity extracted from the slope of the log log plot of mesh sizes vs d values. For this paper, it is assumed that the mesh algorithm being used for this calculation is the box mesh. Automatically computing the mesh dimension of a data set generated from learning agent with speed, accuracy, and robustness is very challenging. A single trajectory provides only a small amount of data, which adds significant noise to the mesh sizes. Agents might do things like fall over and generate extremely short trajectories, or learn a trajectory that "stands in place", which can lead to numerical errors. Finally, every decision is a trade-off between accuracy and speed. Model-free RL is predicated on having a huge number of rollouts to learn from, and we would like for any mesh-dimension quantification algorithm to be fast enough so as to not dominate the total learning time. With these factors in mind, we introduce two box mesh dimensions. The lower mesh dimension does the linear fit, but intentionally errs on the side of including flat parts of the graph, and therefore tends to underestimate the true mesh dimension. We then have the upper mesh dimension, which takes the largest slope in the log log relationship, thus tending to overestimate the true mesh dimension. Neither of these measures are correct, but taken together they can bound the mesh dimension, and as we will see they can be useful on their own.


% !!! may or may not get included
% \subsection{PCA}

% In higher dimensions, the number of mesh points needed to represent data becomes unacceptably large, especially data that has low dimensional structure. To combat this we do a principle component decomposition of a nominal trajectory, while meshing the states, we project each state into the top l PCA values, and crucially, we then scale each projected dimension by the associated normalized singular values. This in effect gives our mesh more resolution in the important dimensions, while allowing less important dimensions to take on a coarser mesh. more formally if the singular value decomposition is given by:


% \begin{equation}
% X = U \Sigma W^{T}
% \end{equation}
% Then our projected coordinates are:

% \begin{equation}
% T = X W \frac{\Sigma}{|\Sigma|}
% \end{equation}


\begin{algorithm}
\begin{algorithmic}[1]
%\KwResult{Mesh Table, Deterministic State Transition Matrix}
\State \textbf{Input:} State set $S$, box size d.
\State \textbf{Output:} Mesh size m.
\State \textbf{Initialize:} Empty hash table M.
 \For{s $\in$ S}
    \State $\bar s  = \text{Normalize(s)} $
    \State key = round($\bar s$ / d)d
    \If{s $\in$ M}
       \State  M[key]++
    \Else
        \State M[key]=1
    \EndIf
\EndFor
\State \textbf{Return:} M
\end{algorithmic}
\caption{Create Box Mesh, see section \ref{sec:boxmsh}}
\end{algorithm}


\begin{algorithm}
\begin{algorithmic}[1]
%\KwResult{Mesh Table, Deterministic State Transition Matrix}
\State \textbf{Input:} State set $S$ 
\State \textbf{Output:} Mesh M.
\State \textbf{Hyperparameters:} scaling factor f,  initial box size $d_{0}$.
\State \textbf{Initialize:} Empty list of mesh sizes H, empty list of d values D.
\State m = Size(CreateBoxMesh(S, $d_{0}$)) 
\State d = $d_{0}$
\State Append m to H, append d to D.
\While{m $<$ size(S)}
    \State d = d/f 
    \State m = Size(CreateBoxMesh(S, d)) 
    \State Prepend m to H, prepend d to D. 
\EndWhile
\While{m $\neq$ 1}
    \State d = d*f  
    \State m = Size(CreateBoxMesh(S,d)) 
    \State Append m to H, append d to D. 
\EndWhile

\State X =  $\log$ d 
\State Y =  -$\log$ m

\textbf{Lower Mesh Dim:} fit Y = gX + b, \textbf{Return:} g \\
\textbf{Upper Mesh Dim:} w = greatest slope in Y over X \textbf{Return:} w

\caption{Compute Box Mesh Dimension, see section \ref{sec:boxdim}}
\label{algo:mesh_dim}
\end{algorithmic}
\end{algorithm}



% Why not use the thing we actually want to reduce? As we've mentioned, measuring the mesh dimension automatically is tricky (again see implementation details in the appendix). But with the faster meshing, it becomes feasible to use the mesh dimension directly, in the same manner we used the variation dimensions before. Here we introduce a second measure of mesh dimesion, called the "conservative mesh dimension". Recall that the normal mesh dimension is the least squares fit of a line to the log log plot of the mesh size vs the box width. The conservative mesh dimension is defined as the steepest slope found on this graph, simple as! This can be seen as an upper bound on the increases that we can expect while decreasing our mesh size. Table \ref{tab:mesh} shows the dimensions and reward after training.


\section{Reinforcement Learning}
The goal of reinforcement learning is to train an agent acting in an environment to maximize some reward function. At every timestep $t \in \mathbb{Z}$, the agent receives the current state $s_{t} \in R^{n}$, uses that to compute an action $a_{t} \in \mathbb{R}^{b}$, and the receives the next state $s_{t+1}$, which is used to calculate a reward $r : \mathbb{R}^{n} \times \mathbb{R}^{m} \times \mathbb{R}^{n} \rightarrow \mathbb{R}$. The objective is to find a policy  $\pi_{\theta}: \mathbb{R}^{n} \rightarrow \mathbb{R}^{m}$

\begin{equation} \argmax_{\theta} \mathop{\mathbb{E}}_{\eta}\left[ \sum_{t=0}^{T}r(s_{t}, a_{t}, s_{t+1}) \right] \end{equation}

Where $\theta \in \mathbb{R}^{d}$ is a set that parameterizes the policy, and $\eta$ is a parameter representing the randomness in the environment. This includes the random initial conditions for episodes.



\subsection{Post Processing Rewards}

In order to influence the dimensionality of the resulting policies, we introduce various postprocessors, which act on the reward signals before passing them to the agent. These obviously modify the problem: in some sense the postprocessed environment is a completely different problem from the original. However our meta-goal is to train agents that achieve reasonable rewards in the base environment, while simultaneously exhibiting reduced dimensionality we are looking for. These postprocessors take the form:
\begin{equation}
R_{*}(\vec{s}, \vec{a}) = \frac{1}{D_{*}(\vec{s})}\sum_{t=0}^{T} r(s_{t}, a_{t}, s_{t+1})
\end{equation}

Where $\vec{s}, \vec{a}$ are understood to be an entire trajectory of state action pairs, and $D_{*}$ is some measure of fractional dimension. Some measures of dimensionality can be inserted here directly (See \ref{sec:var}). However the mesh dimensions computed by algorithm \ref{algo:mesh_dim} require a little more care. We must first define a clipped dimension:

\begin{equation}
D^{c}_{*} = \text{clip}[D_{*}(\vec{s}_{t > Tr}), 1, D_{t}/2)] 
\end{equation}

where $D_{t}$ is the topological dimension, equal to the number of states in the system. $Tr$ is a fixed timestep chosen to exclude the initial transients resulting from a system moving from rest to into a quasi-cyclical ``gait''. In this paper we set $Tr$ = 200 for all experiments. For comparison, the nominal episode length is 1000. The clipping is to ensure that the pathological trajectories that and RL agent sometimes generates don't interfere with the training. It will also clip trajectories that terminate early, to prevent agents learning to fall over immediately to ``game the system''. Half of the topological dimension proved to be a decent upper bound for the worst case dimensionality of each system in practive. The \textbf{mesh dimension postprocessors} use the clipped dimension. Finally, when $D_{*} = 1$ is used, we call the result is the \textbf{identity post processor}, since in this case the total reward is completely unchanged.

\subsection{Environments}
We examine a subset of the popular OpenAI Mujoco locomotion environments introduced in \cite{1606.01540}. In particular, we evaluate our work on HalfCheetah-v2, Hopper-v2, and Walker2d-v2. These environments were chosen because they have a relatively high dimensionality (11-17 DOF), yet we believe can be made feasible for meshing based approaches. The state space consists of all joint / base positions and velocities, with the x (the "forward") position being held out, because we want a policy that is invariant along that dimension.


\begin{figure}
  \centering
  \begin{subfigure}[b]{0.32\textwidth}
    \includegraphics[width=\textwidth]{fig/corl2020/cheetah_crop.png}
    \caption{HalfCheetah-v2}
  \end{subfigure}
  \begin{subfigure}[b]{0.32\textwidth}
    \includegraphics[width=\textwidth]{fig/corl2020/hopper_crop.png}
    \caption{Hopper-v2}
  \end{subfigure}
  \begin{subfigure}[b]{0.32\textwidth}
    \includegraphics[width=\textwidth]{fig/corl2020/walker2d_crop.png}
    \caption{Walker2d-v2}
  \end{subfigure}
  \caption{The Mujoco locomotion environments}
  \label{fig:envs}
\end{figure}


\subsection{Augmented Random Search}

 In \cite{Mania2018} Mania et al introduce Augmented Random Search (ARS) which proved to be efficient and effective on the locomotion tasks. Rather than a neural network, ARS used static linear policies, and compared to most modern reinforcement learning, the algorithm is very straightforward. The algorithm operates directly on the policy weights, each epoch the agent perturbs it's current policy N times, and collects 2N rollouts using the modified policies. The rewards from these rollouts are used to update the current policy weights, repeat until completion. The algorithm is known to have high variance; not all seeds obtain high rewards, but to our knowledge their work in many ways represents the state of the art on these benchmarks. Mania et al introduce several small modifications of the algorithm in their paper, our implementation corresponds to the version they call ARS-V2t.


% \begin{wrapfigure}{R}{0.3\textwidth}
% \centering
% \includegraphics[width=0.25\textwidth]{fig/corl2020/linearz.png}
% \caption{\label{fig:frog1}This is a figure caption.}
% \end{wrapfigure}

% Linearz. We made up a small toy system that exhibited some of the properties we were looking for in our policies. The dynamics are:

% \begin{equation}
%     \dot x = u_{1} , \dot y = u_{2}, \dot z = x
% \end{equation}

% What are the challenges of this environment? it has a dummy dimension, it blows up... 

% \begin{figure}[h!]
%   \centering
%     \includegraphics[width=.25\textwidth]{fig/corl2020/linearz.png}
%     \caption{HalfCheetah}
% \end{figure}

\subsection{Training}
\label{sec:training}

To keep things simple, we wanted to find one set of hyper parameters for all environments and post processors. These parameters were chosen by hand, with the parameters reported in Table 9 from \cite{Mania2018} as a starting point. We tuned until our unprocessed learning achieved satisfactory results across all tasks. Again, ARS is known to have high variance between random seeds, and some seeds never learn to gather a large reward. The parameters we found are able to consistently solve the cheetah and walker; for the hopper, the algorithm learns a policy with high reward around half the time. This seems consistent with the performance reported in \cite{Mania2018}. We train each postprocessor on 10 random seeds, the evaluation metrics are averages over 5 rollouts from each seed, and for the dimension metrics we use extended episodes of length 10,000 to get a more accurate measurement. The reported returns, and the training, both use the normal 1,000 step episodes. We found that the mesh postprocessors were getting very poor performance when trained from a random policy. However, we found that we saw good results when these trials were initialized with a working policy. Therefore we trained agents for 750 epochs without post processing, and used that to initialize the mesh dimension policies. The mesh policies were then trained for an additional 250 epochs and the results reported.
%===============================================================================



\section{Results}
\subsection{Mesh Dimension Postprocessors}

\begin{figure}[!htb]
  \centering
  \begin{subfigure}[b]{0.32\textwidth}
    \includegraphics[width=\textwidth]{fig/corl2020/mesh4_curves/cheetah.png}
    \caption{HalfCheetah}
  \end{subfigure}
  \begin{subfigure}[b]{0.32\textwidth}
    \includegraphics[width=\textwidth]{fig/corl2020/mesh4_curves/hopper.png}
    \caption{Hopper}
  \end{subfigure}
  \begin{subfigure}[b]{0.32\textwidth}
    \includegraphics[width=\textwidth]{fig/corl2020/mesh4_curves/walker.png}
    \caption{Walker}
  \end{subfigure}
  \caption{Reward curves for mesh dimension postproccesor runs.}
  \label{fig:mesh_rews}
\end{figure}

\begin{table}[!htb]
\begin{tabular}{ l|l|l|l|l }
\hline
Environment & Postprocessor 
                  & Lower Mesh Dim.         & Upper Mesh Dim.   & Return \\ 
\hline
\multirow{3}{2.6cm}{HalfCheetah-v2} 
& Identity        & 2.31 $\pm$ 0.71   & 7.34 $\pm$  1.56 &  5469 $\pm$ 823 \\
& Lower Mesh Dim        &  \textbf{0.66 $\pm$ 0.51}   & \textbf{2.55 $\pm$  1.52} &  4962 $\pm$ 598 \\
& Upper Mesh Dim.  & \textbf{1.06 $\pm$ 1.13}  & \textbf{2.83  $\pm$ 1.27} &  4432 $\pm$ 539 \\
\hline
\multirow{3}{2.6cm}{Hopper-v2} 
& Madogram$^{*}$        & 1.62 $\pm$ .27   & 4.68  $\pm$  0.82   &  3461 $\pm$ 119 \\
& Lower Mesh Dim.        & 1.13 $\pm$ .02   & 3.54 $\pm$  0.96   &  2941 $\pm$ 538 \\
& Upper Mesh Dim.  & 1.27 $\pm$ .50   & 2.98 $\pm$  1.48   &  3020 $\pm$ 337 \\
\hline
\multirow{3}{2.6cm}{Walker2d-v2 \\ (walking seeds)$^{**}$}
& Identity        & 2.13 $\pm$  0.31    & 4.62 $\pm$ 1.03  &  3758 $\pm$ 1037  \\
& Lower Mesh Dim.       & 1.21 $\pm$  0.06    & 4.09 $\pm$ 1.03  &  3339 $\pm$ 887   \\
& Upper Mesh Dim. & 1.89 $\pm$  0.42    & 3.10 $\pm$ 0.93  &  3359 $\pm$ 903 \\
\hline
\multirow{3}{2.6cm}{Walker2d-v2 \\ (all seeds)$^{**}$ }
& Identity        & 2.13 $\pm$ 0.31   & 4.62 $\pm$ 1.03    &  3758 $\pm$ 1037 \\
& Lower Mesh Dim.       & 1.04 $\pm$ 0.53   & 4.45 $\pm$ 1.19    &  3034 $\pm$ 1086 \\
& Upper Mesh Dim. & 1.48 $\pm$ 0.67   & 2.27 $\pm$ 0.95    &  2556 $\pm$ 1378 \\
\hline
\end{tabular}
\caption{\label{tab:mesh} Mesh dimensions and returns for trajectories after training. See \ref{sec:training} for details\\
%% \footnotesize{*  Because ARS with our chosen hyper parameters does not consistently produce 10 seeds that perform well on the hopper, we instead use madodiv (see the \ref{sec:var}) for the seed policies.  \\
%% **  See \ref{sec:msh}}
}
\end{table}

For all environments the mesh post processors had a significant impact on the mesh dimensions. It's important to remember here, the dimensions reported represent lower and upper bounds for the actual mesh dimensions. There was also a corresponding and significant decrease in the unprocessed rewards. However with our meta goal of training agents that have acceptable reward but which are more amenable to meshing, this is a more than acceptable trade. In the case of walker, several seeds (4 for the upper dim., 3 for the lower mesh dim.), "forget how to walk", and learn a policy that stands in place. This certainly has a low dimensionality, but is not very useful, to be complete we include statistics from the seeds that learned a gait, and for all 10 seeds, including the standing policies.



\label{sec:msh}

\section{Analysis}

We now examine the learned behavior for one of the more notable policies. By far the most dramatic effect from the tables above was the mesh dimension postprocessors on the cheetah. Both measures of dimension shrunk by 2-4 times.
Figure 7 presents data for this case.

% \begin{figure} (!! there must be a better way to display this figure)
% \centering
% \includegraphics[width=0.4\textwidth]{fig/corl2020/cheetah_analysis/meshes.png}
% \caption{\label{fig:frog2} Half Cheetah Reward Curves.}
% \end{figure}

% \begin{wrapfigure}{L}{0.4\textwidth}
% \centering
% \includegraphics[width=0.4\textwidth]{fig/corl2020/cheetah_analysis/meshes.png}
% \caption{\label{fig:frog2} Half Cheetah Reward Curves.}
% \end{wrapfigure}


\begin{figure}[!htb]
    \centering
    \includegraphics[width=\textwidth]{fig/corl2020/cheetah_mesh_cor.png}        
    \caption{Top: mesh sizes vs log of the box size for the cheetah environment. Lower Left: Every five frames overlaid for the an identity policy on the cheetah. Lower Right: Every five frames of cheetah after the lower mesh dimension training.}
    \label{fig:mesh_anal}
\end{figure}


% \begin{figure}[!htb]
%   \begin{subfigure}[b]{0.88\textwidth}
%     \centering
%     \includegraphics[width=\textwidth]{fig/corl2020/cheetah_mesh_cmp.png}
%     \end{subfigure}
%   \begin{subfigure}[b]{0.4\textwidth}
%     \includegraphics[width=\textwidth]{fig/corl2020/ident_cheetah_crop.png}
%   \end{subfigure}
%  \centering
%   \begin{subfigure}[b]{0.4\textwidth}
%     \includegraphics[width=\textwidth]{fig/corl2020/mdim_cheetah_crop.png}
%   \end{subfigure}
%   \caption{Left: mesh size comparisons across the post processors. middle: Trajectory after training normal ARS for 750 epochs. Right: Trajectory after training 250 additional epochs with the mesh dimension post processor}
% \end{figure}

Toward more intuitively understanding this data, a few comments are worth making, first. We have discussed the mesh dimension rather abstractly so far. In visualizing what this really means, imagine two different gait cycles. In one case, there is a general pattern to the motion, but it wanders in a noisy-looking way, like a ``signature'' that does not quite match up, cycle after cycle.  As motions become closer to being exact limit cycles, there is a more clear pattern of repetition, exactly analogous to re-tracing the same path, again and again, within the state space. Such a more tightly-structured limit cycle nature in turn results in a significantly lower-dimensional set of states being visited. 

We can see from the mesh size curves in Figure 7 that there is an overwhelming difference in the mesh sizes between the lower mesh dimension post processor and the other two. To put this in perspective, before the extra 250 epochs of training, if given a box size of .01, the agent would need a unique mesh point for every single state in the 10,000 state trajectory. After the additional training, the agent can represent all 10,000 points with just 5 mesh points! In this case it appears both agents learned a quasi periodic gait with period 5. In figure \ref{fig:mesh_anal} we present an overlay of the agents rendered every 5 steps. The results show us that the mesh agent has learned an extremely tight limit cycle. It's a bit of a strange limit cycle, being only 5 timesteps long, but nonetheless we think this is interesting and surprising behavior.


% action std = .01
% HalfCheetah-v2:
% |          | mesh dimension | cmesh dimension |
% | identity |           2.33 |            6.78 |
% | mdim_div |           1.50 |            3.69 |

% Hopper-v2:
% |          | mesh dimension | cmesh dimension |
% | madodiv  |           1.64 |            4.52 |
% | mdim_div |           1.44 |            3.60 |

% Walker2d-v2:
% |          | mesh dimension | cmesh dimension |
% | identity |           2.06 |            4.42 |
% | mdim_div |           1.50 |            4.15 |



% State std = .005
% HalfCheetah-v2:
% |          | mesh dimension | cmesh dimension |
% | identity |           2.39 |            6.78 |
% | mdim_div |           1.76 |            3.69 |

% Hopper-v2:
% |          | mesh dimension | cmesh dimension |
% | madodiv  |           2.25 |            4.52 |
% | mdim_div |           1.89 |            4.18 |

% Walker2d-v2:
% |          | mesh dimension | cmesh dimension |
% | identity |           2.29 |            4.59 |
% | mdim_div |           1.89 |            4.36 |



The behavior displayed in figure \ref{fig:mesh_anal} is clearly something that can only happen in a noiseless simulation, so we also measured the mesh dimensions of our policies when subjected to noise during rollouts using the noise values from the robustness experiments (see table \ref{tab:robust})to inform parameter values. The difference in fractal dimension is less pronounced than the no noise case but is still a clear improvement. Furthermore we expect that if we were to add noise at training time, that the learning may be able to find ways to lower the mesh dimension that is more robust to noise.

\begin{table}
\begin{tabular}{ l|l|l|l|l }
\hline
Environment & Postprocessor 
                  & Lower Mesh Dim.         & Upper Mesh Dim.   & Return \\ 
\hline
\multirow{3}{2.6cm}{HalfCheetah-v2} 
& Identity              &  2.38   $\pm$ 0.43   & 6.65 $\pm$  1.90 &  5404 $\pm$ 1015 \\
& Lower Mesh Dim        &  1.51   $\pm$ 0.13   & 3.03  $\pm$  1.09 &  4952 $\pm$ 572 \\
& Upper Mesh Dim.       & 1.76    $\pm$ 0.53      & 3.54  $\pm$  1.27 &  4222 $\pm$ 803 \\
\hline
\multirow{3}{2.6cm}{Hopper-v2} 
& Madogram$^{*}$         & 1.63 $\pm$ .14   & 4.49 $\pm$  0.75   &  3438 $\pm$ 185 \\
& Lower Mesh Dim.        & 1.67 $\pm$ .22   & 3.71 $\pm$  0.89   &  2943 $\pm$ 535 \\
& Upper Mesh Dim.        & 1.64 $\pm$ .16   & 3.01 $\pm$  1.36   &  3019 $\pm$ 337 \\
\hline
\multirow{3}{2.6cm}{Walker2d-v2 \\ (walking seeds)$^{**}$}
& Identity        & 2.13 $\pm$  0.31    & 4.62 $\pm$ 1.03  &  3758 $\pm$ 1037  \\
& Lower Mesh Dim. & 1.83 $\pm$  0.34    & 2.73 $\pm$ 0.75  &  3511 $\pm$ 872   \\
& Upper Mesh Dim. & 1.60 $\pm$  0.33    & 4.01 $\pm$ 1.18  &  3384 $\pm$ 903 \\
\hline
\multirow{3}{2.6cm}{Walker2d-v2 \\ (all seeds)$^{**}$ }
& Identity        & 2.10 $\pm$ 0.34   & 4.42 $\pm$ 1.00    &  3743 $\pm$ 1034 \\
& Lower Mesh Dim. & 1.68 $\pm$ 0.70   & 4.19 $\pm$ 1.25    &  3048 $\pm$ 1071 \\
& Upper Mesh Dim. & 1.48 $\pm$ 0.38   & 2.98 $\pm$ 0.86    &  2558 $\pm$ 1373 \\
\hline
\end{tabular}
\caption{\label{tab:mesh} Mesh dimensions and returns for trajectories subject to zero mean Guassian noise. Standard deviation of .001 and .01 was added to all actions and observations respectively. See \ref{sec:training} for details\\
\footnotesize{Because ARS with our chosen hyper parameters does not consistently produce 10 seeds that perform well on the hopper, we instead use madodiv (see the \ref{sec:var}) for the seed policies.  \\
**  See \ref{sec:msh}}
}
\end{table}

We also performed analysis on the robustness properties of the resulting polices. We tested the failure rate for agents in the presence of noise and disturbances, figure \ref{tab:robust} shows the results. It's worth noting at this point that in practice the lower mesh dimension seems to work better in practice than the upper one. We found that when computing the mesh dimension by hand (by hand fitting a line to a set of carefully obtained mesh size data) that the hand picked value was generally much closer to the lower mesh dimension, at least for the three systems we studied. Training with the lower mesh dimension also resulted in agents that were more robust and achieved higher reward compared to the upper dimension.  

We tested three different cases, adding zero mean Gaussian noise to the actions and observations, and adding a force disturbance to the center of mass of the agents during their rollouts. For the push disturbances we have two parameters, the rate of disturbances, and the mangitude of the force applied. At every step we sample uniformly from [0,1], if the result is less than the rate parameter, then a force is applied at that timestep. The force is applied at a random angle in the xz plane, with the fixed magnitude from the magnitude parameter. For each type of disturbance we did a grid search over the parameters and report the parameter for which the identity post processor failed in around 20 percent of cases. Failure is defined as an early termination of an episode. 


\begin{table}
\centering
\begin{tabular}{ l|l|l|l }

\hline
            & Identity      & Lower mesh dim. & Action std  \\ 
\hline
Cheetah     &          0.24  &           0.05 & .05 \\
Hopper      &          0.19  &           0.10 & .05 \\ 
Walker      &          0.28  &           0.03 & .15 \\
\hline
\hline
            & Identity       & Lower mesh dim. & Observation std  \\ 
\hline
Cheetah     &          0.20  &           0.02 & .005 \\
Hopper      &          0.20  &           0.25 & .02 \\ 
Walker      &          0.18  &           0.10 & .03 \\
\hline
\hline
            & Identity       & Lower mesh dim. & Magnitude, Rate  \\ 
\hline
Cheetah     &          0.21  &           0.03 & 3, .2 \\
Hopper      &          0.17  &           0.10 & 1, .2 \\ 
Walker      &          0.20  &           0.00 & 1, .2 \\
\hline
\end{tabular}
\caption{\label{tab:robust} Failure rates for agents under various noise and push disturbances}
\end{table}


% \section{Related Work}
    
%     !! I think related work belongs at the end, we are mostly comparing to our labs work, and the distinction will be much clearer once the reader knows wtf we are talking about. I think I've seen other papers do this...
%     Outside our lab there really has not been that much work analysing the effects of fractal dimension on reinforcement learning agents. This \cite{Kaygisiz2001} work from back in 2001 does something similar to our meshing approach, and they examine the fractal microstructure of their mesh equivalent. But they make no attempt to actually modulate the fractal structure of the resulting policy. 
    
%     I think I need to mostly talking about our own labs work here? How this differs from it perhaps?


%===============================================================================

% \section{Future Work}

% The obvious next step is to use this to recreate some of the earlier work from our lab. Ideally these techniques will allow our mesh based methods to scale to even higher dimensional systems. There are also improvements that can be made to the meshing algorithm, it's unclear if using a more accurate measure of this during training will aid performance in any way.
    


\section{Conclusion}

In this work, we introduced a technique to influence the fractional dimension of the closed-loop dynamics of a system through the use of novel, dimensionality-based modifications to  the cost functions for reinforcement learning policies. We demonstrate this technique on several benchmark tasks, and we briefly analyze a resulting policy to verify the outcome, demonstrating a much smaller mesh dimension without a large loss in reward or function.

\label{sec:conclusion}
% \begin{center}
% \begin{tabu}{ X[1,l] | X[1,l] | X[1,l] | X[1,l] | X[1,l] }
% Environment & $\alpha$ & $\nu$ & $N$ & $b$ \\
%  \hline
%  \text{HalfCheetah}-v2 & 0.02 & 0.025 & 60 & 20 \\
%  \hline
% \end{tabu}
% \end{center}

%/





\subsection*{Hyper Parameters}

\textbf{ARS:} For all environments $\alpha = .02$, $\sigma = .025$, $N=50$, $b=20$.  \\
\textbf{MeshDim:} f = 1.5, $d_{0}$ = 1e-2



\subsection{Variation Estimators}
\label{sec:var}
As discussed, computing the mesh dimension automatically is fraught with peril, in many practical scenarios. But there are also many other, different metrics one might consider, to give various approximations to the fractional dimension we seek to estimate. Gneiting et al.~\cite{Gneiting2012} compare a number of these estimators, and submit that the variation estimator \cite{Emery2005} offers a very good trade off between speed and robustness. To obtain this estimator, first define the power variation of order p as:

\begin{equation}
P_{p}(X, l) = \frac{1}{(2n-l)}\sum_{i=l}^{n} | X_{i} - X_{i-l}|^{p}
\end{equation}

Then, we define the variation estimator of order p as:

\begin{equation}
Dv_{p}(X) = 2 - \frac{\log P_{p}(X,2) - \log P_{p}(X, 1)}{p\log 2}
\label{eq:var}
\end{equation}

The \textbf{madogram} estimator is the special case of \eqref{eq:var} where p = 1, and the \textbf{variogram} is where p = 2.


\subsection{Variational Postprocessors}


\begin{figure}[!htb]
  \centering
  \begin{subfigure}[b]{0.32\textwidth}
    \includegraphics[width=\textwidth]{fig/corl2020/data17_rews/cheetah.png}
    \caption{HalfCheetah}
  \end{subfigure}
  \begin{subfigure}[b]{0.32\textwidth}
    \includegraphics[width=\textwidth]{fig/corl2020/data17_rews/hopper.png}
    \caption{Hopper}
  \end{subfigure}
  \begin{subfigure}[b]{0.32\textwidth}
    \includegraphics[width=\textwidth]{fig/corl2020/data17_rews/walker.png}
    \caption{Walker}
  \end{subfigure}
  \caption{Reward curves for the variation postprocessors}
  \label{fig:rews}

\end{figure}
\begin{table}
\begin{tabular}{ l|l|l|l|l|l }
\hline
Environment & Postprocessor 
                    & Variogram        & Madogram       & Lower Mesh Dim.   & Return \\ \hline
\multirow{3}{*}{HalfCheetah-v2} 
& Identity          & 1.71 $\pm$ .03   & 1.42 $\pm$ .05 &  2.36 $\pm$ .61 & 5545 $\pm$ 593  \\
& Variogram   & 1.68 $\pm$ .01   & 1.36 $\pm$ .02 &  2.06 $\pm$ .60 & 5136 $\pm$ 851\\
& Madogram    & 1.65 $\pm$ .02   & 1.31 $\pm$ .04 &  2.09 $\pm$ .64 & 5234 $\pm$ 950\\

\hline
\multirow{3}{*}{Hopper-v2} 
& Identity$^{*}$           & 1.61 $\pm$ .14  & 1.22 $\pm$ .28               &  1.03$^{*}$ $\pm$ .71 & 2063 $\pm$ 1052 \\
& Variogram   &  \textbf{1.51 $\pm$ .02}  & \textbf{1.03 $\pm$ .04}   &  1.58 $\pm$ .54 & 3299 $\pm$ 711 \\
& Madogram     & \textbf{1.51 $\pm$ .002}  & \textbf{1.02 $\pm$ .004} &  1.57 $\pm$ .36 & 3449 $\pm$ 146\\
\hline
\multirow{3}{*}{Walker2d-v2} 
& Identity           & 1.68 $\pm$ .35  & 1.36 $\pm$ .71 &              2.14 $\pm$ .29 & 3742 $\pm$ 1038\\
& Variogram    & \textbf{1.54 $\pm$ .07}  & \textbf{1.07 $\pm$ .01} &  1.85 $\pm$ .54 & 3779 $\pm$ 894 \\
& Madogram     & \textbf{1.53 $\pm$ .01}  & \textbf{1.06 $\pm$ .02} &  1.99 $\pm$ .53 & 3414 $\pm$ 1025\\
\hline
\end{tabular}
\caption{\label{tab:dims} Mesh dimensions and returns for trajectories after training. See \ref{sec:training} for details\\
\footnotesize{* This includes policies which learned to "stand still", which lowers the average mesh dimension considerably see discussion}
}
\end{table}

It seems that the variational postprocessors had a modest effect the variational dimension, but that does not seem to correlate to a smaller mesh dimension, despite what our preliminary tests had led us to believe. The hopper and walker did have remarkable consistency in the variation dimensions they found; possibly this could be used to lower the variance in ARS. The fact that the variogram and madogram also got higher performance on the hopper task could support this claim. However without running many more trials and hyper parameter sweeps, that's not a claim that can be substantiated. These experiments show that 1) measures for fractional dimension can be influenced without adversely effecting the reward, and 2) that it is possible for an agent to shrink it's variogram and madogram dimensions without a large impact on its mesh dimension. 



\subsection{Mesh Dimension Examples}

Figure \ref{fig:curves} illustrates two examples of the curves used to compute the mesh dimension. Recall that to compute the mesh dimension, we choose several values for d, the box length, and for each d construct a mesh using that box size. The x axis of these plots represents the log of the box length used, the y axis represents the log of size of the mesh created. For each curve, we display the lower bound and upper bound for the dimension as computed by algorithm 2, as well as several hand fits of the data. We hope that figure \ref{fig:curves}a makes clear what a close to ideal situation looks like, and provide intuition as to why the upper and lower mesh dimension bound the quantity we are trying to measure. Figure \ref{fig:curves}b serves to illustrate some of the problems with making an algorithmic measure of the dimension. There is much less data to work with due to performance constraints, which causes a large amount of noise on the estimate of the mesh dimension. Indeed even fitting this data by hand becomes a challenge and we provide two fits which can both be argued to be "correct". Which of these two represents the quantity we care about depends on the exact system being used and the purpose of the meshes we want to build with the resulting policy.


\begin{figure}
  \centering
  \begin{subfigure}[b]{0.45\textwidth}
    \includegraphics[width=\textwidth]{fig/corl2020/meshdim.png}
    \caption{High resolution curve of a well behaved policy}
  \end{subfigure}
%   \begin{subfigure}[b]{0.32\textwidth}
%     \includegraphics[width=\textwidth]{fig/corl2020/meshdimhand.png}
%     \caption{Hopper-v2}
%   \end{subfigure}
  \begin{subfigure}[b]{0.45\textwidth}
    \includegraphics[width=\textwidth]{fig/corl2020/meshdimreal.png}
    \caption{Run time resolution curve of a typical policy}
  \end{subfigure}
  \caption{Mesh curve and mesh dimension examples}
  \label{fig:curves}
\end{figure}


\subsection{Implementation Details}

For performance reasons, the mesh dimension algorithm does not actually create meshes until the mesh size equals the total data size, but rather until the mesh size is 4/5 the total data size. Figure 8a shows a typical mesh curve, and we can see the long tail of values with mesh sizes close to the maximum value. Not much useful information is gained from this and it is wasting time, so we stop early. We do not place the same limitation on the lower size of the mesh, since typically the mesh size hits one much more rapidly, again figure 8a illustrates this. In addition implement a minimum size for d, set to 1e-9 in this work to avoid numerical errors. 

The normalization done during box creation uses a running mean and standard deviation of all states seen so far during training. These stats are saved and used for evaluation as well, we found that the upper mesh dimension is very sensitive to the normalization used, but that the other metrics where not.



\section{INTRODUCTION}


%is this work improving the control of legged systems or is it moving towards more trusted RL policies
% clearly both, but which to emphasize? should open with whichever one that is.

% Might go with the same "RL is growing in importance because of computers, but how can we trust it" intro that I used for both CDC and CORL

% Not sure we even need to talk about wheeled vehicles 

Legged robots have clear potential to play an important role in our society in the near future. Examples include contact-free delivery during a pandemic, emergency work after an environmental disaster, or as a logistical tool for the military. Legged robots simply expand the reach of robotics when compared to wheeled systems. However, compared to wheeled systems, designing control policies for legged systems is a much more complex task, especially in the presence of disturbances, noise, and unstructured environments.

The increasing availability of massive quantities of computation has led to a resurgence of reinforcement learning (RL) in recent years. RL provides a promising approach for complex, under-actuated, hybrid control problems, such as those involved in designing control for legged locomotion. Recent examples in the context of robotics include controlling a 47 DOF humanoid to navigate a variety of obstacles \cite{heess_emergence_2017}, dexterously manipulating objects with a 24 DOF robotic hand \cite{openai_learning_2018}, and allowing a physical quadruped robot to run \cite{hwangbo_learning_2019}, and recover from falls \cite{lee_robust_2019}.

Despite the obvious promise of RL approaches, several problems need to be resolved before these systems are ready for real world applications. One of the biggest problems is that the resulting policy is typically a complete black box, there are no good ways to make theoretical, or even empirical guarantees about the resulting policies. Prior work has used so called mesh based techniques to this end \cite{Taleledeep}. Broadly, these techniques take a continuous system and approximate it with a discrete set of states. This allows us to model the system as a Markov chain, these systems are arguably easier to reason about, and it opens up a new box of tools we can bring to bear on the problem. For example we can use value iteration to switch between several controllers to improve the robustness \cite{Talelepush} or agility \cite{Byl2017} of the system. We can also perform eigen-analysis on the Markov chain's transition matrix, which provides us insights on the stability of the system \cite{Byl2009}. These techniques could both be used for policy refinement, and/or for verification and analysis of existing policies. 

However, these methods suffer from the "curse of dimensionality", because the number of possible states in our mesh grows exponentially with the degrees of freedom in our system. That is, if we make a change to the volume of continuous space represented by each discrete state, the number of states in our new mesh will grow exponentially with respect to the change in discrete state size. However for virtually all plausible walking controllers, the reachable state space is a small fraction of the total state space. Although the scaling for meshes of the reachable state space also scale exponentially as we increase the mesh resolution, the rate of scaling is typically much smaller. The scaling factor for the reachable mesh can be seen as a \textbf{fractal dimension}, which is elaborated upon in section \ref{sec:fracdim}.

In previous work \cite{Gillen2020ExplicitFractal}, we introduced a modified reward function for on-policy reinforcement learning algorithms. The reward explicitly encourages policies which induce trajectories which have a smaller fractal dimension. It's worth noting that although each individual trajectory was encouraged to have a smaller fractal dimension, this does not obviously extend to properties of the entire reachable state space for the system, which is what was used for the previous mesh based analysis of RL policies. 

In this work we take the next step and construct reachable state space meshes of agents trained with and without our modified reward. Our primary contribution is showing that these modified policies result in significantly smaller reachable meshes for a given box size, and in smaller fractal dimensions for the reachable state space. We then use the modified policies to construct a much finer mesh than would be possible otherwise. We use this mesh to compute a quantity called the mean first passage time (MFPT), and validate the obtained MFPT with Monte Carlo trials. Finally we use our mesh to produce interesting visualizations of failure states, which motivates future work.

\section{BACKGROUND}

In this section we introduce fractional dimensions, meshing, reinforcement learning, and our test environment. The environment is a hopping robot, coupled with a specific reward function. Reinforcement learning is used to train a control policy for this system which attempts to maximize the given reward function. In previous work we used a fractional dimension to modify the reward function, this modified reward results in policies that can be meshed significantly more efficiently. 


\subsection{Meshing and Fractal Dimensions}
\label{sec:fracdim}

% \begin{figure}[h!]
%   \centering
%   \begin{subfigure}[b]{0.45\linewidth}
%     \includegraphics[width=\linewidth]{fig/icra2021//Fractaldimensionexamplebw.png}
%     \caption{Scaling in different dimensions}
%   \end{subfigure}
%   \begin{subfigure}[b]{0.45\linewidth}
%     \includegraphics[width=\linewidth]{fig/icra2021//Snow_Mesh_Example.png}
%     \caption{A non uniform mesh}
%   \end{subfigure}
%   \caption{Image credit: \cite{BrendanRyan/Publicdomain2020}, \cite{Talelepush}.}
%   \label{fig:fracdim}
% \end{figure}

% \begin{wrapfigure}{L}{0.3\textwidth}
% \centering
% \includegraphics[width=0.3\textwidth]{fig/icra2021//HalfCheetahRews.png}
% \caption{\label{fig:frog2} Half Cheetah Reward Curves.}
% \end{wrapfigure}

Let's say we have a continuous set S that we want to approximate by selecting a discrete set M  composed of regions in S. We will call this set M a mesh of our space. Figure \ref{fig:fracdim}(a) shows some examples of this: a line is broken into segments, a square into grid spaces, and so on. The question is: as we increase the resolution of these regions, how many more regions N do we need? Again, Figure \ref{fig:fracdim}(a) shows us some very simple examples. For a D dimensional system, if we go from regions of size d to d/k, then we would expect the number of mesh points to scale as $N \propto k^{D}$. But not all systems will scale like this, as Figure \ref{fig:fracdim}(b)
%and \ref{fig:mdim_example} 
illustrates. Figure \ref{fig:fracdim}(b) is an example of a curve embedded in a two dimensional space. The question of how many mesh points are required must be answered empirically. Going backwards, we can use this relationship to assign a notion of "dimension" to the curve. 

\begin{equation}
    D_{f} = -\lim_{k \rightarrow 0}\frac{\log N(k)}{\log k}. 
    \label{eq:frac_dim}
\end{equation}

This quantity is known as the Minkowski–Bouligand dimension, also called the box counting dimension. This dimension need not be an integer, hence the name "fractional" or "fractal" dimension. This is one of many measures of fractional dimensionality that emerged from the study of fractal geometry. Although these measures were invented to study fractals, they can still be usefully applied to non-fractal sets. For non fractal sets, we use the slope of the log-log relation of mesh sizes to d to compute the dimension, rather than taking a limit. 


% Figure \ref{fig:mdim_example} shows an example of this.

% \begin{figure}[!h] 
% \centering
% \includegraphics[width=.65\linewidth]{fig/icra2021//Snow_LineFitAnn.png}
% \caption{Example of fitting mesh size data to obtain an estimate of the fractal dimension}
% \label{fig:mdim_example}
% \end{figure}

\subsection{Box Meshing}

In this work, we identify any state \textit{s} with a key obtained by: 

\begin{gather}
\nonumber s_{k} = \frac{s - \mu_{s}}{\sigma_{s}} \\ 
\text{key} = \text{round}(\frac{s_{k}}{d_{thr}})d_{thr}.   
\label{eq:key}
\end{gather}

where $\mu_{s}$ and $\sigma_{s}$ are the mean and standard deviation of all the states seen by the policy of interest during training. The round function here performs an element-wise rounding to the nearest integer. We can then use these keys to store mesh points in a hash table. Using this data structure, we can store the mesh compactly, only keeping the points we come across, and lookups are done in constant time. The parameter $d_{thr}$ is called the \textbf{box size}. Geometrically we can think of this operation as dividing the state space into a uniform grid of hypercubes, each with a side length of $d_{thr}$  

\subsection{Reinforcement Learning}

The goal of reinforcement learning is to train an agent acting in an environment to maximize some reward function. At every timestep $t \in \mathbb{Z}$, the agent receives the current state $s_{t} \in R^{n}$, uses that to compute an action $a_{t} \in \mathbb{R}^{b}$, and receives the next state $s_{t+1}$, which is used to calculate a reward $r : \mathbb{R}^{n} \times \mathbb{R}^{m} \times \mathbb{R}^{n} \rightarrow \mathbb{R}$. The objective is to find a policy  $\pi_{\theta}: \mathbb{R}^{n} \rightarrow \mathbb{R}^{m}$ that satisfies: 

\begin{equation} \argmax_{\theta} \mathop{\mathbb{E}}_{\eta}\left[ \sum_{t=0}^{T}r(s_{t}, a_{t}, s_{t+1}) \right]. \end{equation}

Where $\theta \in \mathbb{R}^{d}$ is a set that parameterizes the policy, and $\eta$ is a parameter representing the randomness in the environment. This includes the random initial conditions for episodes.

In \cite{Gillen2020ExplicitFractal}, we introduced a modified reward function:

\begin{equation} 
\argmax_{\theta} \mathop{\mathbb{E}}_{\eta}\left[ \frac{1}{D_{m}(s)}\sum_{t=0}^{T}r(s_{t}, a_{t}, s_{t+1}) \right]
\label{eq:frac_reward}
\end{equation}

where $D_{m}$ is the "lower mesh dimension" explained in detail in \cite{Gillen2020ExplicitFractal}, which is an estimate of equation \ref{eq:frac_dim} dimension for our non fractal set. 

\subsection{Environment}

Our model system is openAI gym's Hopper-v2 environment introduced in \cite{1606.01540}. This environment is part of a popular and standardized set of benchmarking tasks for reinforcement learning algorithms. The system is a 4 link, 6 DOF hopper constrained to travel in the XZ plane, seen in Figure \ref{fig:hopper}. The observation space for the agent has 11 states, the position in the direction of motion is held out, since we seek a policy that is invariant to forward progress. The actions in this case are commanded joint torques. The reward function for this environment is simply forward velocity minus a small penalty to actions. Successful controllers in this environment must execute a dynamic hopping motion to move robot along the x axis as quickly as possible. This is clearly a toy problem, but it captures many of the challenges of legged locomotion. The system is highly non-linear, under-actuated, and must interact with friction and ground contacts to maximize it's reward.  


%Our meta goal here is to construct a controller that can successfully hop, but which is also stable to disturbances and amenable to meshing.

\begin{figure}
\centering
\includegraphics[width=.5\linewidth]{fig/icra2021//hopper_crop.png}
\caption{A render of the hopper system studied in this work.}
\label{fig:hopper}
\end{figure}


% \subsection{Membership functions}
% In all form of meshing, the goal is to represent a continuous set with a discrete approximation. To do this we need two things, one is a function which determines membership of a continuous state in our discrete set. The second is the method with which we obtain the continuous states that we wish to mesh. In most previous work \cite{previous lab work} membership was determined by a nearest neighbor lookup of all the points in the current mesh. 

% \begin{algorithm}
% \SetAlgoLined
% \textbf{Input:} New state $s_{i}$, box size $d_{thr}$, current mesh $\mathbb{M}$. \\
%     $d(s_{i}, M) := \min_{s_{j} \in \mathbb{M}} \sqrt{\sum_{k=0}^{N} (s_{i}(k) - s_{j}(k))^{2}}$ \\
%     \eIf{ $d(s_{i, M}) < d_{th}$}
%         {add $s_{i}$ to set of points represented by argmin !!}
%     {add new mesh point centered at $s_{i}$}
% \caption{Nearest Neighbor Mesh}
% \end{algorithm}

% \begin{algorithm}
% \SetAlgoLined
% \textbf{Input:} New state $s_{i}$, box size $d_{thr}$, current mesh $\mathbb{M}$. \\
%     $ \text{key}_{i} = \text{round}(s_{i}/d_{thr})d_{thr}$ \\
%     \eIf{ $\text{key}_{i} \in M$}
%         {add $s_{i}$ to mesh point argmin !!}
%     {$M = key_{i} \bigcup M$}
% \caption{Hash-Box Mesh}
% \end{algorithm}


\section{MESHING}

In \cite{Gillen2020ExplicitFractal} we used meshes of individual trajectories to calculate fractional dimensions. However the previous work that has used meshing for analysis of RL policies \cite{Taleledeep} instead examines meshes of the reachable state space of a system. In particular \cite{Taleledeep} examines the reachable state space with a fixed control policy subject to a given set of disturbances. We will now outline the process for this style of meshing.  

We are interested in the set of states that our system can transition to with a fixed policy and a given set of push disturbances. We first introduce a failure state to the mesh. The failure state is assumed to be absorbing, once the robot falls it is assumed to stay that way. For our hopper, any state where the COM falls below .7m is considered to have failed, which works well in practice. This is also the failure condition of the environment during training, and therefore the agent is never trained in regions of the state space that satisfy the failure condition. 

In addition to the reachable set of states, we want to construct a state to state transition map. That is, for a given initial state, we wish to know which state we transition to for every disturbance in our disturbance set. It's worth emphasizing that this map is completely deterministic.

To make this concrete, recall that we manifest our mesh as a hash table. The key for any given state is obtained by \ref{eq:key}. When we insert a new key into our hash table, the value we place is a pair with a unique state ID (which is simply the number of keys in the table at the time of insertion), and an initially empty list of all mesh states which are reachable after one step from the key state. This data structure will provide both the reachable set, and the transition mapping.

%Figure \ref{fig:traj} shows an example of this.

For the hopper in particular, the system transitions from its initial standing position to a stable long term hopping gait. After letting the system enter its gait, we start detecting states on the  Poincar\'e section by selecting the state corresponding to the peak of the base link's height in every ballistic phase. These states are then collected as the initial states to seed the mesh with. Throughout this paper, we seed the mesh with trajectories from 10 initial conditions. 

% \begin{figure}[!h]
% \centering
% \includegraphics[width=.8\linewidth]{fig/icra2021//initial_trajectory.png}
% \caption{First 300 time steps of a typical hopper trajectory. Vertical red lines indicate the location of Poincar\'e snapshots.}
% \label{fig:traj}
% \end{figure}


For each snapshot, we initialize the system in the snap-shotted state. For each disturbance in our fixed disturbance set, we simulate the system forward subject to that disturbance. If the system does not fail, then the next Poincar\'e snapshot is captured, this state is then checked for membership in our mesh. If the new state is already in our mesh, then we simply append the new state to the list of states that the initial state can transition to. If the new state is not already in our mesh, then we expand our mesh to include the new state, and append this new state to the transition list of the initial state. If the system does fail, then we simply append the failure state to the transition list of the initial state, and no new state is added to the mesh. 

For every new state added to the mesh, we repeat this process until every state has been explored. Algorithm \ref{algo:createMesh} details this process in pseudo code.

\begin{algorithm}
\SetAlgoLined
%\KwResult{Mesh Table, Deterministic State Transition Matrix}
\textbf{Input:} Initial states $S_{i}$, Disturbance set $D$\\
\textbf{Output:} Mesh M. \\
Q $\leftarrow S_{i}$ (excluding the failure state) \\
\While{Q not empty}{
    pop q from Q \\ 
    \For{d $\in$ D}{
        Initialize system in state q \\
        Run system for one step subject to disturbance d \\
        Obtain final state x \\
        \If{x $\notin$ M}{
            M[x] = List() \\ 
            Push x onto Q
        }
        Append x to M[q]
    }
}
\textbf{Return:} M \\
\caption{createMesh}
\label{algo:createMesh}
\end{algorithm}

\subsection{Stochastic Transition Matrix}

The stochastic transition matrix $\mathbf{T}$ is defined as follows:

\begin{equation}
 \mathbf{T}_{ij} = \text{Pr}(id[n+1] = j \ | \ id[n] = i)
\end{equation}

where $id[n]$ is the index in our mesh data structure of the state at step n. For some intuition, consider the transition matrix as the adjacency matrix for a graph. There is one row/column for every state in our mesh, for a given row i, each entry j is the probability of transitioning from state i to state j. Every row will sum to one, but the sum for each column has no such constraint. After constructing a mesh using algorithm \ref{algo:createMesh}, it is straightforward to create the stochastic transition matrix by iterating through every transition list in our mesh. 

\subsection{Mean First Passage Time}

We wish to use our mesh based methods to quantify the stability of our system. To do this we estimate the average number of steps the agent will take before falling, subject to a given distribution of disturbances. To do this we will use the so called Mean First Passage Time (MFPT) which in this case will describe expected number of footsteps, rather than the number of timesteps to failure. First recall that our assumption is that our failure state is an absorbing state in our Markov chain approximation, and this implies that the largest eigenvalue of \textbf{T} will always be $\lambda_{1} = 1$. In \cite{Byl2009} Byl showed that when the second largest eigenvalue $\lambda_{2}$ is close to unity, the MFPT is approximately equal to:

\begin{equation}
    MFPT \approx \frac{1}{(1-\lambda_{2}).}
    \label{eq:mfpt}
\end{equation}

% \subsection{Mesh Dimension}

% The question is: as we increase the resolution of these regions, how many more regions N do we need? Again, Figure \ref{fig:fracdim}(a) shows us some very simple examples. For a D dimensional system, if we go from regions of size d to d/k, then we would expect the number of mesh points to scale as $N \propto k^{D}$. But not all systems will scale like this, as \ref{fig:fracdim}(b) and \ref{fig:fracdim}(c) illustrate. Figure \ref{fig:fracdim}(b) is an example of a (rather famous) curve embedded in a two dimensional space. The question of how many mesh points are required must be answered empirically. Going backwards, we can use this relationship to assign a notion of "dimension" to the curve. 

% \begin{equation}
%     D_{f} = -\lim_{k \rightarrow 0}\frac{\log N(k)}{\log k} 
% \end{equation}




%\subsection{Reinforcement Learning}

% Not convinced we need this at all?


\section{TRAINING}

In \cite{Mania2018} Mania et al introduce Augmented Random Search (ARS) which proved to be efficient and effective on the locomotion tasks. Rather than a neural network, ARS used static linear policies, and compared to most modern reinforcement learning, the algorithm is very straightforward. The algorithm is known to have high variance; not all seeds obtain high rewards, but to our knowledge their work in many ways represents the state of the art on the Mujoco benchmarks. Mania et al introduce several small modifications of the algorithm in their paper, our implementation corresponds to the version they call ARS-V2t, hyper parameters are provided in the appendix.

The training process is done in episodes, each episode corresponds to 1000 policy evaluations played out in the simulator. At the start of each episode, the system is initialized in a nominal initial condition offset by a small amount of noise added to each state. During each episode we fix a static policy to let the the system evolve under, we collect the observed state, the resulting action, and the resulting reward at each timestep. This information is then used to update the policy for the next episode.

We compare four different sets of agents trained in different conditions, for each training condition we use training runs across 10 different random seeds. As mentioned ARS is a very high variance algorithm, so a common practice is to run many seeds in parallel and choose the highest performing one. The standard environment has two sources of randomness which are set by the random seed. The first is a small amount of noise added to a the nominal initial condition at the beginning of each episode. The second is noise added to the policy parameters as part of the normal ARS training procedure. Using ARS in the unmodified Hopper-v2 environment will be called the \textbf{standard} training procedure. In addition to this, we have a second set of agents which are initialized with the standard training, and then trained for another 250 epochs with the fractal reward function used in equation \ref{eq:frac_reward}, these are called the \textbf{fractal agents}. Using the standard training agents as the initial policies for the fractal reward was also used in \cite{Gillen2020ExplicitFractal}, please see that manuscript for more details. 

In addition to standard training, we repeat this standard / fractal setup but with the addition of a small amount of zero mean Gaussian noise added to both the states and actions at training time. For brevity we will call these the \textbf{Standard noise} and \textbf{Fractal noise} scenarios. Hyper parameters for ARS and noise values are reported in the appendix. 
\section{Results}

\subsection{Mesh Sizes Across All Seeds}
First we wish to compare the reachable state space mesh sizes obtained for these four different training regiments. For this we assume a disturbance profile consisting of 25 pushes equally spaced between -15 and 15 Newtons, applied for 0.01 seconds along the x axis at the apex of each jump. The goal for this particular exercise is to get an idea of the relative mesh sizes among the different agents across box sizes. Table \ref{table:mesh_sizes} shows these results. We can see that across all box sizes, adding noise at run time decreases the mesh sizes slightly, and that adding the fractal reward training decreases the mesh size even further. The combination of adding noise and the fractal reward seems to perform best at reducing the mesh size. 



\begin{table}[!htb]
\centering
\renewcommand{\arraystretch}{1.5}
\begin{tabular}{|l|l|l|l|l|}
\hline 
Training         & $d_{thr} = .4$ & $d_{thr} = .3$ & $d_{thr} = .2$ & $d_{thr} = .1$ \\ \hline \hline
Standard         &    64.9       &     129.0      &   289.2        &  2975.2         \\  \hline
Standard Noise   &    40.7       &     73.3       &   231.6        &  2133.3         \\  \hline
Fractal          &    26.0       &     41.8       &   67.7         &  684.4          \\ \hline
Fractal Noise    &    15.1       &     24.6       &   45.1         &  297.2          \\ 
\hline
\end{tabular}

\caption{Mesh sizes across all seeds for a disturbance profile of 25 pushes. All values are the average mesh size across 10 agents trained with different seeds.}
\label{table:mesh_sizes}
\end{table}



% \begin{table}
% \centering

% \begin{tabular}{|l|l|l|l|l| }
% \hline
% Seed & Iden & Mdim       & Iden Noise & Mdim Noise \\ \hline
% 0    &      &   97319    &            &    1216         \\ 
% 1    &      &   3014     &            &     9774       \\ 
% 2    &      &   479919   &            &     4778       \\ 
% 3    &      &   4041     &            &     1434       \\ 
% 4    &      &   849      &            &     2445       \\ 
% 5    &      &   93067    &            &     3567       \\ 
% 6    &      &   25533    &            &     4667       \\ 
% 7    &      &   15428    &            &     1605       \\ 



% \hline
% \end{tabular}
% \end{table}


\subsection{Larger Meshes}

With the general trend established, we now take the best performing seed from the noisy training for further study. We chose the seed that had the smallest mesh size from both the standard noise and fractal noise agents. 

For this next experiment, we consider a richer distribution of 100 randomly generated push disturbances. These disturbances have a magnitude drawn from a uniform distribution between 5-15 Newtons. This force is applied in the xz plane with an angle drawn from a uniform distribution between 0 and 2$\pi$. The number of forces was chosen by increasing the number of forces sampled until the mesh sizes between two random sets did not change. The magnitude of the pushes was chosen arbitrarily, in principle one can use these methods for any distribution of disturbance they expect their robot to encounter during operation. 
% \begin{figure}[!h]
% \centering
% \includegraphics[width=.4\linewidth]{fig/icra2021//noise.png}
% \caption{Noise profile for detailed experiments on single seeds, magnitudes range between 5 and 15 newton meters}
% \label{fig:noise}
% \end{figure}

We then construct meshes for different box sizes. For each agent we construct 10 meshes. We vary the box size between 0.1 and 0.01 for the fractal noise agent. For the standard noise agent we instead vary the box size between 0.1 and 0.02 because the mesh sizes for the standard agent were proving to be too large at the smaller box sizes. Figure \ref{fig:mesh_cmp} shows the comparison, We can see clearly that at the very least, the exponential blowup in mesh size starts at much more accurate mesh resolutions for the fractal agent.

\begin{figure}[!htb]
\centering
\includegraphics[width=.7\linewidth]{fig/icra2021//mesh_size_cmp.png}
\caption{Mesh sizes for the top performing standard noise and fractal noise agents.}
\label{fig:mesh_cmp}
\end{figure}

We are also interested in the exponential scaling factor in the mesh size as the box gets smaller, which is captured by the fractal dimension discussed in section \ref{sec:fracdim}. As mentioned before, in previous work our modified reward signal resulted in agents with a smaller fractal dimension with respect to individual trajectories. We now ask if this carries over to meshes of the reachable state space obtained by the procedure from algorithm \ref{algo:createMesh}. Table \ref{table:mesh_dims} shows the results, we can see that indeed, the fractal training does seem to reduce the mesh dimensionality for the reachable state space meshes. 



\begin{table}[!h]
\renewcommand{\arraystretch}{1.5}
\begin{tabular}{|l|l|l|}
\hline 
Training              & Trajectory Mesh Dim.    &  Reachable Mesh Dim. \\ \hline \hline
Standard Noise        &    1.38                 &      3.83     \\  \hline
Fractal Noise         &    1.16                 &      3.16     \\  \hline
\end{tabular}

\caption{Mesh dimensions for the best performing seed from the standard with noise training, and the fractal with noise training, given the same disturbance profile of 100 pushes. For reference the state space for our system has 12 dimensions.}
\label{table:mesh_dims}
\end{table}

\subsection{Validating the Mean First Passage Time}

We emphasize that the reward function for the hopper environment is simply to move forward with the highest velocity possible, no attempts were made to make the system robust to disturbances. Perhaps because of this, the mean first passage time for these systems are relatively small, on the order of 100 foot steps. For this small number of steps, we can validate the mean first passage time with Monte Carlo trials. It's worth noting that the eigen estimate of the mean first passage time is much more valuable for more robust systems. This is because this estimate becomes more accurate as the system becomes more stable, and because the cost of calculating the MFPT with Monte Carlo trials grows much more expensive for more stable systems. In previous works \cite{OguzSaglam2015} it was used to quantify robustness for systems with a MFPT as high as $10^{15}$. 

To do this, we compare the mean first passage time as estimated by equation \ref{eq:mfpt} to the value computed by looking at many Monte Carlo rollouts. For the rollouts we apply a random action drawn from the same distribution described above. Instead of sampling 100 pushes though we sample a new push every time we need a new disturbance. During the rollouts we still apply the push at the apex height of the ballistic phase.

Figure \ref{fig:mdim_monte} shows the convergence of the MFPT as we expand the size of the mesh, and compares it to the mean steps to failure obtained with Monte Carlo trials. We can see that it does look like the MFPT is converging to the Monte Carlo result. Although at the largest mesh we tried, the eigen analysis gives an estimate of 110.2 steps to failure, while the Monte Carlo trials tell us that an average of 85 steps are taken before failure. It's worth noting that the distribution of failure times has a large variance with a standard deviation of 80 steps.


\begin{figure}[!h]
\centering
\includegraphics[width=.75\linewidth]{fig/icra2021//mdim_mfpt.png}
\caption{Estimated mean first passage time computed from \ref{eq:mfpt} compared to a Monte Carlo estimate. The blue dashed line and shaded region are the mean and standard deviation of the steps to failure for 2500 Monte Carlo rollouts.}
\label{fig:mdim_monte}
\end{figure}


% \begin{figure}[!h]
% \centering
% \includegraphics[width=.65\linewidth]{fig/icra2021//iden_mfpt.png}
% \caption{Estimated mean first passage time from eigenvalues of the stochastic transition matrix for different box sizes. The blue dashed line is the mean steps to failure for 2500 Monte Carlo rollouts,
% The shaded region indicates the standard deviation of the Monte Carlo trials}
% \label{fig:iden_monte}
% \end{figure}


\subsection{High Resolution Mesh}

We now use the fractal agent and construct an even more accurate mesh. Figure \ref{fig:mdimT} show the sparsity pattern for the state transition matrix for the fractal noise agent with a box size of 0.005. Recall that in the process for creating the mesh, we start with a small number initial seed states. After that every new state that we add is added in order we find them to the mesh. So if we are expanding state \# 2, and there are currently 100 states in the mesh, if we transition to an unseen state, that state will be labeled \# 101. So although it may seem like it is not possible for states in the top right quadrant to visit states later in the mesh, this is really an artifact of how we construct our mesh and label our points. 

% \begin{figure}
% \centering
% \includegraphics[width=.65\linewidth]{fig/icra2021//iden_noise_25.png}
% \caption{Sparsity pattern for an agent trained with the with the standard reward and noise added at training time. All non zero values are shown with equal size and coloration. The matrix can be interpreted as follows, each row represents the state that you are starting from at time t, and each collum with an entry represents a probability that you will transition to the corresponding collumn at time t+1}
% \label{fig:idenT}
% \end{figure}


\begin{figure}[!htb]
\centering
\includegraphics[width=.7\linewidth]{fig/icra2021//sT2.png}
\caption{Visualization of the stochastic transition matrix for the top performing fractal noise agent. All non zero values are shown with equal size and coloration. Recall that each entry in $T_{ij}$ tell us the probability of transitioning to state j after one step if we start in state i.}
\label{fig:mdimT}
\end{figure}

%(!!! this whole paragraph needs some work, might remove)
We note that there are a smaller set of states that make up most of the transitions. In fact we can see from Figure \ref{fig:mass_clump} that 20\% of the states in our mesh account for about 90\% of all transitions seen during the mesh construction.

\begin{figure}[!htb]
\centering
\includegraphics[width=.7\linewidth]{fig/icra2021//pmass.png}
\caption{Cumulative sum of probability mass excluding the failure state. We take the sum of each column of T, and sort it in descending order, then report the cumulative sum of probability. Each point on the curve tells us that x\% of states make up y\% of all state transitions.}
\label{fig:mass_clump}
\end{figure}

One of the advantages of having a discrete set of states is that it opens up new tools and visualizations, for example we can apply Principle Component Analysis (PCA). Figure \ref{fig:pca_side} shows a projection of our mesh states on the top 3 principle components. We note that these three states account for more than 97\% of the variance, we also note that our analysis reveals that states in red are where 99\% of all failures occur. The visualization reveals that at least in PCA space, all the trouble states are clustered in one spot. A promising direction for future work is to introduce a policy refinement step that attempts to avoid these states. Additionally, if we were designing a real robot this may give us insights into design changes that could be made. 


\begin{figure}
\centering
\includegraphics[width=\linewidth]{fig/icra2021//mdim_pca_side.png}
\caption{View of the first 3 principle components of the mesh for a fractal noise policy.}
\label{fig:pca_side}
\end{figure}

% \begin{figure}
% \centering
% \includegraphics[width=1\linewidth]{fig/icra2021//mdim_pca_top.png}
% \caption{}
% \label{fig:pca_top}
% \end{figure}


% \begin{figure}
% \centering
% \includegraphics[width=.65\linewidth]{fig/icra2021//iden_mfpt.png}
% \caption{}
% \label{fig:mdimT}
% \end{figure}

% Can still add these... it does maybe speak to some of the strengths meshing can provide. 

% \begin{figure}
% \centering
% \includegraphics[width=\linewidth]{fig/icra2021//mdim1_fail.png}
% \caption{Placeholder}
% \label{fig:mass_clump}
% \end{figure}





\section{CONCLUSIONS}

In this work, we apply previously developed tools that create discrete meshes for the reachable state space of a system. These tools were applied to policies obtained with a modified reinforcement learning reward function which was previously shown to encourage small mesh dimensions for individual trajectories not subject to any disturbances. We showed that these modified policies have a smaller average reachable mesh size across all random seeds for coarse meshes and a small number of disturbances. We then showed a clear difference in mesh sizes and mesh dimensions for the top performing seeds on a richer set of disturbances and finer mesh sizes. We also validated our use of the MFPT as a tool by comparing it to Monte Carlo trials. Finally, we constructed a high fidelity mesh at a resolution that would not have been feasible with standard ARS policies. In addition, we created visualizations with this mesh that revealed insights about the contracting nature of the policy, and which point to future applications of this approach. Taken together, these results show two things. First, it further validates the utility of the fractal dimension reward, which we have shown transfers it's desirable quality of having a more compact state space to a setting with external disturbances. These results are also a credit to the mesh based tools, because it shows that the fractal training can be used to extend the reach of these tools to higher dimensional systems or higher resolution meshes than would have otherwise been possible.  


\section*{APPENDIX}

\subsection*{Hyper Parameters}
\textbf{ARS:} (from \cite{Mania2018}) $\alpha = 0.02$, $\sigma = 0.025$, $N=50$, $b=20$.  \\
\textbf{MeshDim:} (from \cite{Gillen2020ExplicitFractal}) f = 1.5, $d_{0}$ = 1e-2


\subsection*{Noise During Training}
Zero mean Gaussian noise with std = 0.01 added to policy actions before being passed to the environment, for reference all actions from the policy are between -1 and 1.  Zero mean Gaussian noise with std = 0.001 added to observations before being passed to the policy.  \\ 

% \subsection*{Implementation Details} Recall that our mesh uses a hash table with a key constructed by \ref{eq:key}. During the meshing process we must reconstruct the original state s. One option is simply to invert equation \ref{eq:key}. Geometrically, this can be viewed as taking the centroid of the box of continuous space that the mesh state represents. For small mesh sizes this is an unimportant detail, however for some of the larger boxes used in constructing table \ref{table:mesh_sizes}, we found that taking the centroid resulted in states with significant ground penetration, which caused instability in the simulator. To alleviate this, we store the first state which resulted in adding a new box to the mesh as the representative for that box. This representative is then used when initializing the system to explore the mesh point.



%\section*{ACKNOWLEDGMENT}




%%%%%%%%%%%%%%%%%%%%%%%%%%%%%%%%%%%%%%%%%%%%%%%%%
\section{Abstract}
Researchers have demonstrated that Deep Reinforcement Learning (DRL) is a powerful tool for finding policies that perform well on complex robotic systems. However, these policies are often unpredictable and can induce highly variable behavior when evaluated with only slightly different initial conditions. Training considerations constrain DRL algorithm designs in that most algorithms must use stochastic policies during training. The resulting policy used during deployment, however, can and frequently is a deterministic one that uses the Maximum Likelihood Action (MLA) at each step. In this work, we show that a direct random search is very effective at fine-tuning DRL policies by directly optimizing them using deterministic rollouts. We illustrate this across a large collection of reinforcement learning environments, using a wide variety of policies obtained from different algorithms. Our results show that this method yields more consistent and higher performing agents on the environments we tested. Furthermore, we demonstrate how this method can be used to extend our previous work on shrinking the dimensionality of the reachable state space of closed-loop systems run under Deep Neural Network (DNN) policies. 

% "will likely" sounds speculative/prediction-oriented--maybe "often has" if it's more a statement that exploration noise is usually different from the robotic system?
% / feels informal--would "or" make sense here?
% could be helpful to specify "this method" in the last sentence! 



%\blfootnote{Code hosted at: \url{github.com/sgillen/policy_refinement}}
\section{Introduction}



% Maybe should shrink the example section. 
In recent years, researchers have leveraged Deep Reinforcement Learning (DRL) to solve a wide variety of continuous control problems. Examples include problems from the computer graphics community, in which DRL has been used for physics-based character animation~\cite{2018-TOG-deepMimic}, and a wide variety of complex robotic tasks. This paper focuses primarily on applications in robotics. Continuous control problems in the context of robotics include controlling a 47 degree-of-freedom (DOF) humanoid to navigate various obstacles~\cite{heess_emergence_2017}, dexterously manipulating objects with a 24 DOF robotic hand~\cite{openai_learning_2018}, training the quadrupedal ANYmal robot to recover from falls~\cite{lee_robust_2019}, and teaching the bipedal Cassie robot to navigate stairs blindly~\cite{siekmann2021blind}. These problems are all high dimensional, nonlinear, and underactuated, and they all involve complex contact sequences with the environments, which makes them very challenging for more traditional control design. Traditional model-based control techniques are still very effective---arguably, Boston Dynamics still represents the state-of-the-art for legged locomotion in robotics, for example. However, these approaches require hundreds of expert person hours to develop each new controller. DRL attempts to automate at least some aspects of this challenging controller development process. There are already examples of learned policies outperforming ones hand-designed by experts~\cite{hwangbo_learning_2019}, and with the ever-continued growth and availability of computational power, there is good reason to believe these learning methods will continue performing better and becoming easier to use.

%\katie{citation here?} \sean{not sure what to cite since BD hasn't published since like big dog AFAIK} \katie{good point. I guess just a video, but we have a LOT of refs already, so maybe nevermind}. My vote is to ignore it unless maybe 


But, of course, there are significant drawbacks to these model-free approaches. While Deep Neural Networks (DNNs) are very powerful, they also need to acquire a lot of data during training. This contributes to DRL being very sample inefficient, meaning that many interactions with the environment are required in order to find a good policy.  As a result, most training for robotic systems must be done in simulation, where the environment can be parellelized and run thousands of times faster than real time. Transfer learning is often required to adapt such policies so that they work for real-world hardware. Doing so effectively remains an important, open problem. Furthermore, modern DRL algorithms can be difficult to implement, as small implementation details can change performance dramatically~\cite{engstrom2020implementation}, which motivates our additional focus in reducing the observed variability in performance of closed-loop policies from DRL.


% As there is often this requirement to fine-tune any policy by running it on the physical system on which it will be deployed, we focus throughout, algorithmically, on this fine-tuning step. 


DRL policies are almost always stochastic in nature. During training almost all the common DRL algorithms either add exploration noise to the actions, or learn a probability distribution from which to sample at training time. This might be, for example, a simple Gaussian distribution, the more sophisticated Ornstein-Uhlenbeck correlated noise process in the case of Deep Deterministic Policy Gradient DDPG~\cite{lillicrap_continuous_2015}), or, in the case of DQN, a random selection of sub-optimal actions~\cite{mnih2015humanlevel}. However, when policies are deployed or evaluated, one typically uses a deterministic policy by taking what we will call the Maximum Likelihood Action (MLA). 

In \cite{Mania2018}, the authors show that a simple Augmented Random Search (ARS) over linear functions was competitive with deep reinforcement learning across a standard suite of benchmark tasks. Furthermore, this algorithm is simple and, in the cases the authors tested, around fifteen times more sample efficient than the best-performing DRL baseline. Despite these advantages, the simplicity of the policy class limits the environments to which it can currently be applied. 


%\sean{Maybe just cut the example? I think most people will fill in their own. I was thinking like, we know a linear classifier can't even learn the XOR function, so what hope does it have on say classifying terrain based on pixels? etc..., open to other non vision ones maybe...}
%for example, it is clear that any vision based task would be impossible with this approach. \katie{(Um... not clear to me? Can we add a teensy bit more? b/c you need a gradient for each pixel?)}

% "In [10] feels like a non-intuitive way to begin a sentence to me!

%And we think that applying this algorithm to deep networks loses some potency. Let's take one of the locomotion benchmarks as an example, the Bullet version of hopper has 16 observations and 3 actions for a total of 48 learnable parameters. If we look at the default parameters for this same environments for td3, the actor network alone has 128003 parameters, and has an equally sized critic network to learn as well.  

In this work, we show that a slightly modified version of this random search can be applied directly to DNNs for fine tuning, without any apparent loss in sample efficiency. The simplicity of this approach has several advantages. The first is that it does not appear to be very sensitive to hyper-parameter settings. We are able to use a single set of parameters for all the results obtained in this paper, across a dozen environments, and with most systems being tested for six different initial policies each obtained from a different DRL algorithm. Second, we avoid some of the previously mentioned problems stemming from the complexity and fragility of modern DRL algorithms. Finally, %we appear to 
our data thus far indicate that we seem to achieve essentially the same sample efficiency seen in ARS, despite operating over much larger parameterizations.

We show that our proposed method of fine tuning leads to modest increases in reward and substantial improvements to consistency in performance for DRL agents across a large set of RL environments. In addition, we also show that we can also use this fine-tuning method to extend previous work of ours involving an extra dimensionality term in the reward~\cite{Gillen2020ExplicitFractal}.

The rest of this paper is laid out as follows. First, we introduce the problem statement, the algorithms and environments used, and implementation details for the training. We then present results obtained from using this policy refinement approach across a collection of continuous-control RL environments. We also perform some analysis on how often fall events occur for a benchmark bipedal walker where the existing DRL baselines are particularly prone to failure. After this, we present results of using this method to train with additional, dimensionality based reward terms in order to show that we are able to extend our previous work to DNN policy classes. Finally, we demonstrate the approach on a Panda arm simulation environment, where our approach leads to considerably smoother policies that avoid unwanted jitter, without any environment specific or algorithm specific tuning or reward shaping.  

\begin{table*}[!hbtp]
\label{tab:1}
\centering
\begin{tabular}{p{2cm} p{2cm}p{1.5cm}p{1.5cm}p{1.5cm}p{1.5cm}p{1.5cm}p{1.5cm}}
Environment &        &  A2C &                       PPO &                      DDPG &    TD3 & SAC &                       TQC \\
\bottomrule
\end{tabular}

\begin{tabular}{p{2cm}p{2cm}p{1.5cm}p{1.5cm}p{1.5cm}p{1.5cm}p{1.5cm}p{1.5cm}}
\toprule
\multirow{2}{*}{MountainCar}
&Baseline Return &  91 ± 0.2 &   88 ± 2.3 &  93 ± 0.0 &  93 ± 0.1 &  94 ± 1.3 &  \textbf{67 ± 43.8} \\
&Tuned Return    &  92 ± 0.1 &  96 ± 17.0 &  94 ± 0.4 &  94 ± 0.2 &  95 ± 1.1 &   \textbf{96 ± 0.9} \\
\end{tabular}


\begin{tabular}{p{2cm}p{2cm}p{1.5cm}p{1.5cm}p{1.5cm}p{1.5cm}p{1.5cm}p{1.5cm}}
\toprule
\multirow{2}{*}{LunarLander}
&Baseline Return &  61 ± 137.3 &  273 ± 30.5 &  216 ± 100.0 &  \textbf{205 ± 86.7} &  259 ± 67.8 &  279 ± 28.6 \\
&Tuned Return    &  160 ± 126.1 &  275 ± 32.4 &   249 ± 68.5 &  \textbf{257 ± 20.1} &   283 ± 18.1 &  286 ± 17.7 \\
\end{tabular}

\begin{tabular}{p{2cm}p{2cm}p{1.5cm}p{1.5cm}p{1.5cm}p{1.5cm}p{1.5cm}p{1.5cm}}
\toprule
\multirow{2}{*}{BoxWalker}
&Baseline Return &  \textbf{296 ± 27.0} &  \textbf{220 ± 122.4} &  \textbf{217 ± 127.4} &  \textbf{302 ± 65.1} &  \textbf{289 ± 66.0} &  \textbf{326 ± 58.2} \\
&TunedReturn &  \textbf{313 ± 0.7} &   \textbf{325 ± 0.7} &   \textbf{281 ± 54.1} &  \textbf{334 ± 0.6} &    \textbf{321 ± 1.0} &   \textbf{344 ± 0.3} \\
\bottomrule
\end{tabular}


\begin{tabular}{p{2cm}p{2cm}p{1.5cm}p{1.5cm}p{1.5cm}p{1.5cm}p{1.5cm}p{1.5cm}}
\multirow{2}{*}{BoxWalkerHard}
&Baseline Return &   99 ± 129.3 &  137 ± 119.4 &  N/A &  -92 ± 16.3 &  16 ± 104.2 &  238 ± 102.0 \\
&Tuned Return    &  109 ± 121.0 &  137 ± 119.7 &  N/A &   -23 ± 5.2 &   44 ± 86.4 &  242 ± 107.6 \\
\bottomrule
\end{tabular}


% \begin{tabular}{p{2cm}p{2cm}p{1.5cm}p{1.5cm}p{1.5cm}p{1.5cm}p{1.5cm}p{1.5cm}}
% \multirow{2}{*}{Pendulum}
% &Baseline Return &  -230 ± 99.2 &  -171 ± 127.4 &  -157 ± 99.3 &  -110 ± 83.5 &  -116 ± 116.1 &  -144 ± 45.9 \\
% &TunedReturn &  -229 ± 98.5 &  -149 ± 109.1 &  -152 ± 87.6 &  -110 ± 83.8 &   -101 ± 97.1 &  -144 ± 45.8 \\
% \bottomrule
% \end{tabular}


\begin{tabular}{p{2cm}p{2cm}p{1.5cm}p{1.5cm}p{1.5cm}p{1.5cm}p{1.5cm}p{1.5cm}}
\multirow{2}{*}{Walker2D}
&Baseline Return &   785 ± 389.2 &  2108 ± 16.0 &  \textbf{1432 ± 720.1} &  \textbf{2218 ± 194.6} &  \textbf{2290 ± 34.8} &  \textbf{2540 ± 557.6} \\
&Tuned Return &  913 ± 269.3 &  2250 ± 194.1 &   \textbf{1896 ± 375.7} &   \textbf{2411 ± 7.5} &   \textbf{2413 ± 13.6} &    \textbf{2812 ± 8.8} \\
\bottomrule
\end{tabular}

\begin{tabular}{p{2cm}p{2cm}p{1.5cm}p{1.5cm}p{1.5cm}p{1.5cm}p{1.5cm}p{1.5cm}}
\multirow{2}{*}{HalfCheetah}
&Baseline Return &  2109 ± 36.3 &  2938 ± 53.7 &  2064 ± 198.7 &  2820 ± 21.0 &  2792 ± 10.9 &  \textbf{3676 ± 16.7} \\
&Tuned Return    &  2211 ± 35.9 &  3000 ± 42.3 &  2264 ± 133.1 &  2928 ± 15.4 &   2883 ± 6.9 &  \textbf{3802 ± 11.9} \\
\bottomrule
\end{tabular}

\begin{tabular}{p{2cm}p{2cm}p{1.5cm}p{1.5cm}p{1.5cm}p{1.5cm}p{1.5cm}p{1.5cm}}
\multirow{2}{*}{Hopper}
&Baseline Return &   \textbf{834 ± 343.3} &  \textbf{2523 ± 383.5} &   \textbf{1179 ± 453.1} &   2681 ± 27.2 &  2602 ± 205.2 &  \textbf{2631 ± 329.7} \\
&Tuned Return &  \textbf{1643 ± 204.1} &    \textbf{2633 ± 91.0} &  \textbf{2379 ± 341.6} &  2749 ± 337.1 &   2706 ± 96.7 &   \textbf{2782 ± 20.7} \\
\bottomrule
\end{tabular}


\begin{tabular}{p{2cm}p{2cm}p{1.5cm}p{1.5cm}p{1.5cm}p{1.5cm}p{1.5cm}p{1.5cm}}
\multirow{2}{*}{Ant}
&Baseline Return &  2502 ± 25.4 &   2869 ± 72.7 &  2365 ± 212.5 &  \textbf{3268 ± 288.8} &  3096 ± 31.3 &  \textbf{3478 ± 24.0} \\
&Tuned Return    &  2679 ± 28.4 &  2897 ± 157.0 &   2424 ± 86.7 &   \textbf{3391 ± 24.8} &  3206 ± 18.0 &   \textbf{3654 ± 21.7} \\
\bottomrule
\end{tabular}

% \begin{tabular}{p{2cm}p{2cm}p{1.5cm}p{1.5cm}p{1.5cm}p{1.5cm}p{1.5cm}p{1.5cm}}
% \multirow{2}{*}{Reacher}
% &Baseline Return &  14 ± 8.5 &  17 ± 8.4 &  16 ± 10.7 &  12 ± 5.6 &  14 ± 11.2 &  15 ± 9.0 \\
% &TunedReturn &  14 ± 8.5 &  17 ± 8.0 &  15 ± 11.3 &  12 ± 5.7 &  13 ± 11.2 &  15 ± 9.1 \\
% \bottomrule
% \end{tabular}}

\caption{Average Return $\pm$ standard deviation before and after fine tuning}

%\vspace{-5mm}
\end{table*}
\begin{table}[hb]
\center
\small
\begin{tabular}{ llll }
\toprule
Env. & Algo. 
                  & Fail \% Before         &  Fail \% After \\ 
\midrule
\multirow{4}{*}{Walker}

% & PPO        & 2.31 $\pm$ 0.71   & 7.34 $\pm$  1.56 \\
% & TD3   & \textbf{1.06 $\pm$ 1.13}  & \textbf{2.83  $\pm$ 1.27} \\ 
% & SAC   & \textbf{1.06 $\pm$ 1.13}  & \textbf{2.83  $\pm$ 1.27} \\
% & TQC   & \textbf{1.06 $\pm$ 1.13}  & \textbf{2.83  $\pm$ 1.27} \\
% & A2C   & \textbf{1.06 $\pm$ 1.13}  & \textbf{2.83  $\pm$ 1.27} \\
% & DDPG   & \textbf{1.06 $\pm$ 1.13}  & \textbf{2.83  $\pm$ 1.27}\\

& A2C   &       19.33   &       12.00 \\
& PPO   &       0.00    &       0.33 \\
& DDPG  &       42.67   &       4.00 \\
& TD3   &       12.33   &       1.67 \\
& SAC   &       2.33    &       0.67 \\
& TQC   &       5.67    &       0.00 \\

\bottomrule
\end{tabular}
\caption{\label{tab:falls} Measured early termination events before and after the fine tuning process \\
}
%Put here to reduce too much white space after your table 
\end{table}



\section{Methods}


\subsection{Reinforcement Learning}

The goal of reinforcement learning is to train an agent acting in an environment to maximize some reward function. At every timestep $t \in \mathbb{Z}$, the agent receives the current state $s_{t} \in \mathbb{R}^{n}$, uses that to compute an action $a_{t} \in \mathbb{R}^{b}$, and then receives the next state $s_{t+1}$, which is used to calculate a reward $r : \mathbb{R}^{n} \times \mathbb{R}^{m} \times \mathbb{R}^{n} \rightarrow \mathbb{R}$. The objective is to find a policy  $\pi_{\theta}: \mathbb{R}^{n} \rightarrow \mathbb{R}^{m}$

\begin{equation} \argmax_{\theta} \mathop{\mathbb{E}}_{\eta}\left[ \sum_{t=0}^{T}r(s_{t}, a_{t}, s_{t+1}) \right] \end{equation}
where $\theta \in \mathbb{R}^{d}$ is a set that parameterizes the policy, and $\eta$ is a parameter representing the randomness in the environment. We will call the sum of rewards obtained during an episode a return. 

\subsection{Direct Policy Search}
\label{sec:search}

We start by noting that there are many names for what we are calling direct policy search, as it is at least 50 years old and has been rediscovered by a variety of different optimization communities. Algorithm~\ref{algo:brs} outlines our particular version of it. In essence the random search chooses 2n candidate policies at each step by adding zero-mean Gaussian noise to the current policy parameters. These candidate policies are used to perform rollouts, and the reward for each rollout is recorded. These rewards are then used in the update step for the policy.

In~\cite{Mania2018}, the authors show that this direct policy search is competitive with DRL. Specifically, they add a number of "tricks" to the basic algorithm and call their resulting approach the Augmented Random Search (ARS). We add our own set of tricks in this work. First, we keep ARS's update step, where the step size is divided by the standard deviation of returns obtained. Second, we maintain the normalization functions learned by the DRL algorithms we are tuning. This differs from algorithm to algorithm, but usually it involves normalization of the data using statistics of the observation that is seen during training, followed by a clipping operation. We also found that it is important to be careful with the random seeds used for rollouts, and so for each policy pair $\theta \pm \delta_{i}$ we ensured that the environment used the same seed. This was particularly important for environments with a wide distribution of initial conditions. Finally, we found performance was slightly improved by using a linear schedule for step size and exploration noise. 

One advantage of this method that we have found is that it is not very sensitive to hyper-parameters. For every result presented in this paper, we deliberately used the same parameters: 200 update steps, $n=64$, $\alpha=[0.02, .002]$, and $\sigma=[0.025, 0.0025]$, which were chosen using the parameters used in~\cite{Mania2018} as a starting point. Ignoring for a second that 200 update steps is in fact more than is necessary for most environments, this implies that our method takes 25600 rollouts to train. In simulation with parallel rollouts, this is completed in a matter of minutes using a Ryzen 3900x. For the Panda arm environments that we will discuss in more detail later, this would correspond to about 14 hours of real robot time, and we suspect this time could be brought down considerably by tuning the hyper-parameters specifically for sample efficiency. 

We also note that we ran experiments where we train only a subset of the neural network parameters, which would make the number of trainable parameters comparable to the linear policies used in~\cite{Mania2018}. During these experiments we found the results were slightly inferior to training on the entire network, and that the sample efficiency, measured by number of updates required to reach a given reward threshold, was almost exactly the same. 




\begin{algorithm}
\caption{Direct Policy Search}\label{algo:brs}
\begin{algorithmic}[1]
\Require Policy $\pi$ with trainable parameters $\theta$
\Require Hyper-parameters - $\alpha$ $\sigma$ $n$
\State Sample $\bf{\delta} = [\delta_{1}, ..., \delta_{n}]$ from $\mathcal{N}(0, \sigma)^{\text{n x }\abs{\theta}}$
\State $\theta^{*}  = [\theta - \delta_{1}, ..., \theta - \delta_{n}, \theta + \delta_{1}, ..., \theta + \delta_{n}] $
\For{$\theta_{i}$ in $\theta^{*}$}
    \State Do rollout with policy $\pi_{\theta_{i}}$, using the MLA
    \State Collect sum of rewards $R_{i}$. 
\EndFor
\State $ \theta^{+} = \theta + \frac{\alpha}{n \sigma_{R}}\sum_{i=0}^{n} (R_{i} - R_{i+n})\delta_{i} $ 
\end{algorithmic}
\end{algorithm}


\subsection{Environments}

\label{sec:envs}

We examine a number of popular benchmarking environments from the RL community. The environments all conform to the OpenAI Gym API introduced in~\cite{1606.01540}. For ease of reference, we will refer to each environment by the ID it has in the Gym registry. MountainCarContinuous-v0, LunarLandarContinuous-v2, BipedalWalker-v3, and BipedalWalkerHardCore-v3 are all standard continuous control environments included with the base Gym environments. To the best of our knowledge these environments are not meant to be physically realistic. We also study a collection of locomotion environments implemented in PyBullet. The locomotion environments were created by~\cite{galloudec2021multigoal} and are maintained by the Bullet Physics team~\cite{coumans2020}. In this work we study  HalfCheetahBulletEnv-v0, HopperBulletEnv-v0, Walker2DBulletEnv-v0, and AntBulletEnv-v0. All of these environments are simulated legged robots. Agents take joint angles and velocities as input states, and compute joint torques as actions. The reward functions are designed to encourage agents to walk forward as fast as possible. It may be worth noting that these are inspired by OpenAI's popular Mujoco environments, though the Bullet versions are considerably heavier and impose more realistic torque limits, which makes them a bit more challenging for RL algorithms. In the second half of this paper, we study a set of environments based on a 7DOF Franka Emika Panda arm~\cite{galloudec2021multigoal}. These environments are made difficult both by their complexity and the fact that they use a sparse reward structure. As an example of these aspects, consider the PandaPickAndPlace-v1 environment, in which the arm must pick up a block somewhere in its workplace and bring it to a randomized goal state. The agent recieves a reward of -1 everywhere except when the block has reached the goal state. 



%In \cite{andrychowicz2018hindsight}, the authors showed that Hindsight Experience Replay (HER) combined with an off policy algorithm was able to solve this, in this work we use HER with TQC
% \begin{figure}[!htb]
%     \centering
%     \includegraphics[width=\linewidth]{figs/fetch-traj.png}
%     \caption{}
%     \label{fig:tracj}
% \end{figure}



\subsection{Pre Trained Agents}

We use the Stable Baselines 3 Zoo \cite{rl-zoo3} \cite{stable-baselines3} for a collection of pretrained agents with tuned hyper parameters. The Zoo provides agents for Truncated Quantile Critics (TQC),  Soft Actor Critic (SAC), Proximal Policy Optimization (PPO), Asynchronous Actor Critic (A2C), Deep Deterministic Policy Gradients (DDPG), and Twin Delayed Deep Deterministic policy gradient (TD3) \cite{kuznetsov2020controlling} \cite{haarnoja2018soft} \cite{schulman2017proximal} \cite{mnih2016asynchronous} \cite{lillicrap_continuous_2015} \cite{fujimoto2018addressing}. In all examples, the policies are deep neural networks and the exact architecture has been tuned by the Zoo maintainers to have reasonable performance for each environment algorithm pair. We use these policies to initialize the values of $\theta$ in Algorithm~\ref{algo:brs}. %1. 


\bgroup
\def\arraystretch{1.1}%
\begin{table*}[h]
\centering
\begin{tabular}{p{2cm} p{2cm}p{1.5cm}p{1.5cm}p{1.5cm}p{1.5cm}p{1.5cm}p{1.5cm}}
Environment &        &  A2C &                       PPO &                      DDPG &    TD3 & SAC &                       TQC \\
\bottomrule
\end{tabular}
\begin{tabular}{p{2cm}p{2cm}p{1.5cm}p{1.5cm}p{1.5cm}p{1.5cm}p{1.5cm}p{1.5cm}}
\toprule
\multirow{2}{*}{Walker2D}
&Baseline Dim.   &      2.55 ± 0.6 &      3.45 ± 0.4 &       5.54 ± 0.5 &      \textbf{6.09 ± 1.6} &       5.96 ± 1.6 &       \textbf{5.36 ± 0.5} \\
&Tuned  Dim.   &      1.21 ± 0.3 &      2.35 ± 0.2 &       3.82 ± 0.3 &       \textbf{3.72 ± 0.3} &       3.85 ± 0.5 &       \textbf{3.71 ± 0.2} \\
\cline{2-8}
&Baseline Return &   785 ± 389.2 &  2108 ± 16.0 &  1432 ± 720.1 &  \textbf{2218 ± 194.6} &  2290 ± 34.8 &   \textbf{2540 ± 557.6} \\
& Tuned  Return &    997 ± 2.2 &  2024 ± 10.1 &   1961 ± 12.5 &   \textbf{2152 ± 27.6}   &   2269 ± 13.3 &   \textbf{2562 ± 12.6} \\
\bottomrule
\end{tabular}

\begin{tabular}{p{2cm}p{2cm}p{1.5cm}p{1.5cm}p{1.5cm}p{1.5cm}p{1.5cm}p{1.5cm}}
\multirow{2}{*}{HalfCheetah}
&Baseline Dim.   &   3.19 ± 0.3 &   3.35 ± 0.2 &   \textbf{4.31 ± 0.4} &   \textbf{5.17 ± 0.3} &   4.83 ± 0.3 &   3.65 ± 0.2 \\
&Tuned  Dim.   &    2.4 ± 0.2 &   2.54 ± 0.2 &   \textbf{3.01 ± 0.3} &   \textbf{2.76 ± 0.3} &   3.46 ± 0.3 &   2.56 ± 0.2 \\
\cline{2-8}
&Baseline Return &  2109 ± 36.3 &  2938 ± 53.7 &  \textbf{2064 ± 198.7} &  2820 ± 21.0 &  2792 ± 10.9 &  3676 ± 16.7 \\
&Tuned  Return &  2137 ± 22.3 &  2778 ± 27.5 &  \textbf{2594 ± 41.9} &  2697 ± 13.1 &  2658 ± 12.1 &   3606 ± 7.2 \\
\bottomrule
\end{tabular}

\begin{tabular}{p{2cm}p{2cm}p{1.5cm}p{1.5cm}p{1.5cm}p{1.5cm}p{1.5cm}p{1.5cm}}
\multirow{2}{*}{Hopper}
&Baseline Dim.   &   2.85 ± 0.5 &    3.16 ± 0.5 &  3.67 ± 0.5 &   3.76 ± 0.4 &   \textbf{5.12 ± 0.3} &   \textbf{5.12 ± 0.3} \\
&Tuned  Dim.   &   2.24 ± 0.1 &    2.31 ± 0.2 &   3.12 ± 0.1  &   2.74 ± 0.1 &    \textbf{2.7 ± 0.2} &    \textbf{2.3 ± 0.1} \\
\cline{2-8}
&Baseline Return &  \textbf{834 ± 343.3} &  \textbf{2523 ± 383.5} &  \textbf{1179 ± 453.1} &   2681 ± 27.2 &  \textbf{2602 ± 205.2} &  \textbf{2631 ± 329.7} \\
&Tuned  Return &  \textbf{2072 ± 12.4} &   \textbf{2559 ± 26.0} &  \textbf{2641 ± 39.2} &   2763 ± 7.4 &   \textbf{2687 ± 8.1} &  \textbf{2547 ± 10.4} \\
\bottomrule
\end{tabular}


\begin{tabular}{p{2cm}p{2cm}p{1.5cm}p{1.5cm}p{1.5cm}p{1.5cm}p{1.5cm}p{1.5cm}}
\multirow{2}{*}{Ant}
&Baseline Dim.   &   2.65 ± 0.2 &   3.91 ± 0.6 &   7.14 ± 0.4 &    5.76 ± 0.2 &   7.17 ± 0.3 &    5.25 ± 0.3 \\
&Tuned  Dim.   &   2.15 ± 0.2 &   3.11 ± 0.1 &    6.87 ± 0.3 &    4.29 ± 0.4 &   3.35 ± 0.2 &    3.39 ± 0.2 \\
\cline{2-8}
&Baseline Return &  2502 ± 25.4 &   2869 ± 72.7 &  \textbf{2365 ± 212.5} &  3268 ± 288.8 &  3096 ± 31.3 &  3478 ± 24.0 \\
&Tuned  Return &  2527 ± 13.5 &  2817 ± 26.8 &  \textbf{2498 ± 42.9} &  3330 ± 100.1 &   2854 ± 8.0 &    3488 ± 3.4 \\
\bottomrule
\end{tabular}

% \begin{tabular}{p{2cm}p{2cm}p{1.5cm}p{1.5cm}p{1.5cm}p{1.5cm}p{1.5cm}p{1.5cm}}
% \multirow{2}{*}{Reacher}
% &Baseline Return &  14 ± 8.5 &  17 ± 8.4 &  16 ± 10.7 &  12 ± 5.6 &  14 ± 11.2 &  15 ± 9.0 \\
% &Tuned  Return &  14 ± 8.5 &  17 ± 8.0 &  15 ± 11.3 &  12 ± 5.7 &  13 ± 11.2 &  15 ± 9.1 \\
% \bottomrule
% \end{tabular}

\caption{Returns and Dimensionality after Fine Tuning with an extra dimensionality reward term}
\label{tab:mesh}
%\vspace{-8mm}

\end{table*}


\section{Results}




First we examine the results of using our direct policy search for policy fine-tuning of a large set of environments and initial policies, using the parameters from Section~\ref{sec:search}. We compare the mean and standard deviation of returns before and after our fine-tuning process. In both cases, the policies are evaluated deterministically by using the MLA at each step, the only randomness in the system is from the initial condition at the start of each episode, which is drawn from the same distribution seen during training. Each agent is evaluated with 100 Monte Carlo trials. 

The results are presented in Table~I. %\ref{tab:1}. % 1. 
In almost all cases, we see at least a modest improvement to average return. Recalling that an even more fundamental goal in this work is to reduce variability, also note that many cases resulted in a substantial decrease in the variance of the return, as desired. This suggests that our fine-tuning process is effective both for squeezing extra performance out of a trained DRL agent and also for reducing the variability of those agents. 


We also examine the robustness of these policies. We note that the baseline agents, even with no noise added and using the deterministic policy evaluation, will experience failure events from some particular initial conditions. Here we define failure as any "early termination" event from the environment. In the case of Walker2DBulletEnv-v0, the environment automatically terminates early if a non-foot link contacts the ground or if the simulation determines that a fall is imminent due to its center of mass location or body orientation. To test robustness, we sample 300 initial conditions and evaluate the policies both before and after our refinement step. We present the results from the Walker2D system because it had the highest failure rate across all baselines algorithms. We can see in Table~\ref{tab:falls} that DDPG, for example, failed in about 42\% of cases before the refinement process and in around 4\% afterwards. The other algorithms show improvement as well. In the case of TQC, we went from failing about 5\% of the time to not detecting any failure events during the 300 trials.




\section{Mesh Dimensions}


In previous work \cite{Saglam-RSS-14}, we introduced what we call a ``mesh dimension'' as an component to reward functions for reinforcement learning agents. Informally, agents typically operate in relatively high dimensional state spaces. However in practice they will often only move along a comparatively lower-dimensional manifold within that full space. That is, although motions are not completely synchronized over time, they demonstrate quite a bit of coordination among joints. By eye, such a gait-like coordination is often quite apparent. The mesh dimension attempts to identify this dimensionality reduction quantifiably. It estimates the dimensionality of the reachable state space of the closed-loop system, and, for those familiar with the term, it is very closely related to a ``fractal dimension''. %\katie{(yikes - could put a ref here, too... but I know we have too many already, perhaps...)}

In another line of prior work \cite{Gillen2020ExplicitFractal}, we showed that ARS was able to train linear policies on environments which were modified to include this mesh dimension reward. This had a number of desirable qualities including finding very precise periodic gaits in some cases, and it improved robustness to push disturbances and sensor noise. In that work training used a lower and upper bound of the estimated dimensionality; in this work we train on the average of those two bounds. 

We experimented with several ways to incorporate this measure of dimensionality into the reward function, including both a linear and quadratic combination with the original reward. While these methods worked to some extent, they required fairly precise manual tuning of coefficients. Somewhat surprisingly, we found that simply taking the product of the original reward multiplied by the reciprocal of the dimension estimate $D$ was an effective reward that required no manual tuning:
\begin{equation}
\label{eq:Rr}
R^{r} = \frac{ \sum_{t=0}^{T}r(s_{t}, a_{t}, s_{t+1})}{D} .
\end{equation}

One caveat is that this only works for environments with positive rewards. For negative returns, however, as in the case of the Panda environments, we can simply take the product instead:
\begin{equation}
\label{eq:Rp}
 R^{p} = D \sum_{t=0}^{T}r(s_{t}, a_{t}, s_{t+1}) .
\end{equation}
We found that these rewards successfully gave the agents a signal to optimize, leading to significant reductions in dimensionality without any significant degradation in performance over the original reward. 


% Basically we can take a continuous autonomous system, decide on a fixed, discrete set of disturbances, and create a markov chain approximation of the system. This has a lot of interesting applications. One must pick a resolution in state space when they discretize, and smaller resolutions will make models which are closer to the real system. The mesh dimension tells us how fast the size of our markov chain grows as we increase our resolution. 

% Formally, there is some math, and if I need space I can go into that right here.


% If we had used PPO \cite{schulman2017proximal} for example, the mesh dimension would always be being raised artificially by the stochastic nature of the policy, which stops it from following this particular reward signal. 

% We might do a simple product:

% \[ R  \Omega \]

% Which is what we did in \cite{Gillen2020ExplicitFractal}, and we found it to work well despite the fact that it really should not have. 

% More common in multi objective optimizationm, we can try linear combination:

% \[ \alpha R + \beta \Omega \]

% Or maybe if we're feeling really spicy:

% \[ |\alpha(R - R^{*})^{n} + \beta(\Omega - \Omega^{*})^{n} | ^{1/n}  \]

% To find the $R^{*}$ we use the maximum reward obtained during training, but we are also careful  



\section{Mesh Dimension Results}

\subsection{Locomotion Environments}

First, we present results for fine-tuning with the post-processed reward from Equation~\ref{eq:Rp}. In this work we train DNNs on the more difficult Bullet environments. Results are shown in Table~III. All agents are evaluated deterministically here, using the MLA. Each entry in the Table for mean and standard deviation for the return and for the estimated dimensionality is calculated based on 100 Monte Carlo Trials. We see that the dimensionality (``Dim.'') in most environments is decreased quite drastically, particularly for the off-policy algorithms (DDPG, TD3). As before, this process also seems to decrease the variability of the return, perhaps even more reliably than without the dimension reward. We believe this shows that our previous results can be extended to DNNs, which greatly expands the scope of problems they can be applied to.






\subsection{Panda Arm Environments}


% Describe these here or in the environments section?
% Discuss difficulties with smaller reward

We present data here for a set of environments utilizing a Panda arm, introduced earlier in Section~\ref{sec:envs}. These environments present several challenging problems for an Emika Franka Panda arm. The PandaReach task is the most straightforward. Here, the goal is for the arm to reach a given point in task space. For PandaPush, the arm mush push a block along the floor to a desired location. In PandaSlide, the robot must grab a block and and bring it to a desired point on the ground. Finally, PandaPickAndPlace requires the arm to pick up a block and keep its grip on it while attempting to reach a point in space. We found that merely fine-tuning the action network with our random search did not improve performance significantly, though to be fair, none of the algorithms in the Zoo are able to solve this environment without Hindsight Experience Replay (HER)~\cite{andrychowicz2018hindsight}. 


\begin{table}[!ht]
\begin{tabular}{lllll}
\toprule
          Environment & Base Dim. & Our  Dim. & Base. Return & Our  Return \\
\midrule
        PandaReach &    2.73 ± 0.7 &    2.28 ± 0.5 &        -2 ± 0.6 &        -1 ± 0.7 \\
 Pick\&Place &    1.63 ± 0.3 &    1.61 ± 0.5 &        -6 ± 2.6 &      -11 ± 13.3 \\
         PandaPush &    1.91 ± 0.5 &    1.68 ± 0.3 &        -6 ± 2.7 &        -7 ± 3.0 \\
        PandaSlide &    1.89 ± 0.4 &    1.53 ± 0.3 &       -22 ± 7.1 &      -41 ± 12.4 \\
\bottomrule
\end{tabular}
\caption{Dimensionality and Returns before and after fine tuning for the panda environments}
\end{table}




We did however also apply our dimensionality reward signal to this environment using our fine tuning process. We show the resulting dimensionality, and the returns in terms of the original reward function are shown in Table~IV. We observed a modest decrease in the dimensionality, accompanied by some decrease in the original return. Again, this decrease in reward is not unexpected, as we are after all trained on modified reward function. In addition to this, despite the Table data that suggests perhaps only a small change in behavior was observed, we noticed a significant beneficial change in the qualitative behavior of the robot. Figure~\ref{fig:panda} shows a stark example of this. In the PandaReach environment, the baseline agent is able to get its end effector into the target region, however it exhibits undesirable shaking behavior which the reward function does not punish. Our agent is able to achieve the same effect with a smooth motion. Note that both agents received exactly the same reward for the episodes we show,i.e. that despite the jittering deviations in the end effector, it remain in the goal region. 

\begin{figure}[!htb]
    \centering
    \includegraphics[width=\linewidth]{fig/icra2022/panda_smoother.png}
    \caption{End effector positions and velocities for a policy roll out on PandaReach before and after fine tuning with the mesh dimension reward}
    \label{fig:panda}
\end{figure}
\egroup

% \begin{figure}[!htb]
%     \centering
%     \includegraphics[width=.8\linewidth]{figs/fetch-push.png}
%     \caption{}
%     \label{fig:just_because}
% \end{figure}


%probably don't need but .. sideways table https://arxiv.org/pdf/1604.06778.pd


\section{Discussion and Related Work}

% \textcolor{red}{want to strike a fairly modest tone here, not "our method is the best thing and no other alternatives come close", but more  "if your problem is like X, this other approach is probably better, if your problem is like Y, our method provides a good tool, emphasize broad range of applications (graphics, demos, human computer interaction, benchmarks? general robotics, etc."}

It's worth discussing alternatives to our method for fine-tuning. The most similar work to ours that we have found is~\cite{pmlr-v100-xie20a}. There, the authors take a similar approach in that they decouple the algorithms used for exploration versus exploitation. We agree with their conclusion that this decoupling brings advantages on its own, regardless of the methods used for the fine-tuning exploitation. In some sense their work is doing the opposite of our approach, however, in that after using a gradient-free evolutionary strategy for exploration, it then uses DRL for exploitation (rather than for exploration). While the approaches are quite distinct, they are also in fact likely compatible, in that they could actually be combined. It is easy to imagine a pipeline using their gradient-free method for broad exploration, followed by DRL for initial exploitation, with our random search added for the final fine-tuning stage of an algorithm. 

There are also many small tricks and improvements found in DRL algorithms that aim to achieve similar results to what we've shown. One example is to decrease the SGD/Adam/RMSProp step size as training goes on. This is an effective method, and indeed our own method uses a linear schedule for the step size. However, the algorithms we are using as a baseline were already using this approach as well, and we still saw improvements in performance with the additional of our fine tuning process. 

Entropy regularization / penalties are another toolset available, which can also be put on a schedule. These can encourage an agent to use a wide distribution of actions initially and then gradually narrow this down as training continues. Again though, most of the algorithms used as a baseline (PPO, SAC, TQC) have some form of this already, and our method is still able to improve on them. 

We could try curriculum learning, meaning that the reward function could change by design as training goes on. We believe this is likely most effective when one has a lot of domain knowledge of the task, and when being applied to tasks that are too difficult for the algorithm to learn initially. For example, the authors of~\cite{xie2021policy} use this approach when controlling Cassie. This approach works well for them because they are able to engineer a reward that led to the desired behavior. 


\section{Conclusion}


We have presented a method that can fine tune policies obtained from DRL algorithms by optimizing directly using the MLA. We showed that performance compared to a baseline was improved considerably on a large set of standard benchmarking tasks. Of more particular note, the variability of episode returns was decreased significantly on many of the environments we tested as well. For the system on which we also quantified failure rates (i.e., for the biped Walker), this lower variability was also accompanied by significantly fewer early termination events compared to the baseline. We hypothesize that this increased robustness is, quite plausibly, due to the dimensionality reduction. (That hypothesis is in fact why we performed these experiments, of course.) However, any conclusions on correspondence remain a topic for further investigation. 

We also showed that this method allows us to expand our previous work on adding dimensionality metrics to the reward function of RL agents to DNNs as well, which greatly expands the scope of problems it can be applied to. We demonstrated this on a set of locomotion environments and also on a challenging set of Panda arm environments with sparse reward structures. We showed that for the case of the Panda our approach achieved significantly less jitter, and arguably more visually pleasing (and mechanically desirable)
%I had so much jitter in my masters-thesis segway-type robot during testing that the pinion on a Maxon motor popped off - rendering the motor useless. It is no joke that jitter will break robots. Hopefully, at least some people reading this will "get" that... 
motion than the baseline, without any environment specific reward shaping, or manual adjustment of any parameters. 

We believe versatility and simplicity are major strengths of this approach. Policies obtained from any kind of DRL algorithm can be tuned in this way, and the method seems to require very little manual tweaking. The potential applications for this method are broad. Engineers designing robots which are public facing or that interact with humans may find it useful to employ policies that make their robots motions smoothing and thereby easier for humans to predict. There are applications outside of robotics as well. Physics-based character animation may also benefit from more consistently behaving policies, and DRL is also popular for video game AI, which is another area where the improved consistency of this method may prove desirable. 

Finally, we will end with a discussion on the broader impacts and future directions of this work. DRL has been an exciting and promising paradigm for robotic control for some time now, but it has yet to be widely adopted by industry. This is largely because it is difficult to trust a DNN controller, and deploying a poorly understood controller can be expensive and dangerous. By itself, we think the fine tuning method we've introduced can help make DRL policies more effective and reliable, however we also think that the lower dimensional policies can unlock even more tools to aid with this. With lower dimensional polices, we open the possibility to develop methods to perform numerical estimates of a variety of controls-based metrics, such as rates of contraction (Lyapunov exponents), identification of dangerous regions in state space (outside a stochastic separatrix for a basin of attraction), and/or expected (conservative) distributions of failure rate. All of these are promising directions towards safer and more reliable DRL based control, and we anticipate that our method brings us closer to realizing them for useful, real world, robotic systems.   


% There is also the problem that training DRL, even in well behaved virtual environments, is notoriously fickle. Small changes in hyper parameters, or seemingly insignificant implementation details can lead to drastic shifts and learned behavior, and often agents initialized with different weights or using a different random seed will end up with completely different behavior. We therefore believe that using simple and easy to understand tools should be employed wherever possible, and we think our proposed method reflects that philosophy.

%\katie{Finally we'll note that, at a 20,000-foot viewpoint, this work provides a modest step in the direction of several general goal we have within our research lab. Deep reinforcement learning remains quite mysterious and, thereby, risky to date, in that quantifiable metrics for robustness are arguably pretty lacking at present. Better connections between model-based control theory successes of the past and modern machine learning would be great to have. The notion of assuming a system can be treated as ``close to linear'' if it's near desired operating conditions, which is the basis of LQR after all, has some qualitative similarities with our hunch that linear policies from ARS may be found once the system is already controlled to be close to reasonable operating conditions (i.e., a local minimum), due to a policy from an initial DRL exploration stage. Our work here gives evidence that this general idea is perhaps worth pursuing further, at any rate. Additionally, if we can successfully come up with control policies that limit the dimensionality of the reachable state space of a closed-loop system, this could potentially make it less daunting to develop future methods to perform numerical estimates of a variety of controls-based metrics, such as rates of contraction (Lyapunov exponents), identification of dangerous regions in state space (outside a stochastic separatrix for a basin of attraction), and/or expected (conservative) distributions of failure rates, i.e., it lends a glimmer of hope in chipping away at the ``curse of dimensionality''.}

\medskip




\chapter{Differentiable Simulators}








% paper title
\title{Cross Entropy Analytic Policy Gradients: A Reinforcement Learning Algorithm for Continuous Control In Differentiable Physics Simulations}

% You will get a Paper-ID when submitting a pdf file to the conference system
%\author{Author Names Omitted for Anonymous Review. Paper-ID [add your ID here]}

%\author{\authorblockN{Michael Shell}
%\authorblockA{School of Electrical and\\Computer Engineering\\
%Georgia Institute of Technology\\
%Atlanta, Georgia 30332--0250\\
%Email: mshell@ece.gatech.edu}
%\and
%\authorblockN{Homer Simpson}
%\authorblockA{Twentieth Century Fox\\
%Springfield, USA\\
%Email: homer@thesimpsons.com}
%\and
%\authorblockN{James Kirk\\ and Montgomery Scott}
%\authorblockA{Starfleet Academy\\
%San Francisco, California 96678-2391\\
%Telephone: (800) 555--1212\\
%Fax: (888) 555--1212}}


% avoiding spaces at the end of the author lines is not a problem with
% conference papers because we don't use \thanks or \IEEEmembership


% for over three affiliations, or if they all won't fit within the width
% of the page, use this alternative format:
% 
%\author{\authorblockN{Michael Shell\authorrefmark{1},
%Homer Simpson\authorrefmark{2},
%James Kirk\authorrefmark{3}, 
%Montgomery Scott\authorrefmark{3} and
%Eldon Tyrell\authorrefmark{4}}
%\authorblockA{\authorrefmark{1}School of Electrical and Computer Engineering\\
%Georgia Institute of Technology,
%Atlanta, Georgia 30332--0250\\ Email: mshell@ece.gatech.edu}
%\authorblockA{\authorrefmark{2}Twentieth Century Fox, Springfield, USA\\
%Email: homer@thesimpsons.com}
%\authorblockA{\authorrefmark{3}Starfleet Academy, San Francisco, California 96678-2391\\
%Telephone: (800) 555--1212, Fax: (888) 555--1212}
%\authorblockA{\authorrefmark{4}Tyrell Inc., 123 Replicant Street, Los Angeles, California 90210--4321}}

\section{Abstract}
New advances in simulation have resulted in physics engines with fully differentiable dynamics. This opens up new possibilities for reinforcement learning, namely the possibility of using analytic gradients for policy search. However these gradients have proven to be very challenging to use in practice, and the performance of algorithms based on these gradients lag behind algorithms which do not make use of these algorithms at all. In this paper we introduce an algorithm we call Cross Entropy Analytic Policy Gradients, an algorithm which improves on the performance of Analytic Policy Gradients, and which outperforms state of the art model free reinforcement learning for certain systems which model free algorithms historically struggle on. This work represents a step towards reinforcement learning algorithms that can productively use analytic gradients for policy search. 

\section{Introduction}

Computer simulation has become an indispensable tool for researchers and engineers of robotic systems for design, control, and verification. Some recent advances in robotic control, especially the field of model free reinforcement learning, are completely dependent on accurate and fast simulations of robotic systems. This is because RL is extremely sample inefficient. This stems partially from the fact that we cannot directly take the gradients of the reward function with respect to controller parameters. This fact is due to the fact in the commonly used simulators for reinforcement learning, like MuJoco \cite{todorov2012mujoco} or Bullet \cite{coumans2020}, are not differentiable, we are unable to take gradients through the simulators themselves. This is of course also true for physical robotic systems, and thus reinforcement learning must thus rely on various approximations of the true gradients, like finite differences or REINFORCE \cite{williams1992simple}

However in recent years, fully differentiable physics simulators have started to emerge \cite{hu2020difftaichi} \cite{heiden2021neuralsim} \cite{brax2021github}. These simulators all offer analytic gradients through the simulator itself, using automatic differentiation. These simulators have a number of applications and advantages over traditional simulators. For example, it is possible to use data from a physical system to modify simulation to better match, in what can be termed real2sim transfer. Furthermore the nature of these simulators allow them to be run on hardware accelerators, for example GPU or TPUs. And of course they also provide analytic gradients. That is to say, we can define a reward function, which takes as input the simulator state, and which outputs a scalar value which is to be maximized. We can then find the analytic gradient of this reward function with respect to policy parameters.


Let's be clear, in the context reinforcement learning, the goal is to train an agent, acting in an environment, to maximize some reward function. The environment is a discrete time dynamical system described by state $s_{t} \in \mathbb{R}^{n}$ and the current action $a_{t} \in \mathbb{R}^{b}$. an evolution function $f: \mathbb{R}^{n} \times \mathbb{R}^{b} \rightarrow \mathbb{R}^{n}$ takes as input the current state and action, and outputs the state at time t+1:



\begin{equation}
s_{t+1}= f(s_{t},a_{t})
\end{equation} 

The controller is the function we are learning $g: \mathbb{R}^{n} \times \mathbb{R}^{\norm{\theta}} \rightarrow \mathbb{R}^{m}$ such that:

\begin{equation}
a_{t} = g(s_{t}, \theta)
\end{equation} 

The goal is to maximize a reward function $r : \mathbb{R}^{n} \times \mathbb{R}^{m} \times \mathbb{R}^{n} \rightarrow \mathbb{R}$. We consider the finite time undiscounted reward case. The objective function then is:

\begin{equation} 
R(\theta) =  \sum_{t=0}^{T}r(s_{t}, a_{t}, s_{t+1}) 
\end{equation}

% \begin{equation} \argmax_{\theta} \mathop{\mathbb{E}}_{\eta}\left[ \sum_{t=0}^{T}r(s_{t}, a_{t}, s_{t+1}) \right] \end{equation}

% Where $\nu$ is a parameter representing the randomness of the environment or controller. It may come from a randomized initial condition, from sensor noise, or from a stochastic policy. 

One simple method to do this is simple gradient descent, that is we do a simple iterative method updated by taking the gradient of R with respect to the policy parameters:

\begin{equation}
    \label{eq:vapg}
    \theta^{+} = \theta + \alpha \nabla_\theta R(\theta)
\end{equation}

Typically, gradients for f, the environment has been unavailable. This makes it impossible to compute the gradient $\nabla_{\theta}R(\theta)$ directly. There are several ways around this, one can use zeroth order methods like Evolutionary Strategies (ES) \cite{salimans2017evolution}, or Augmented Random Search (ARS) \cite{Mania2018} which don't require any gradients at all. If you have the gradients for g, for example in the case of a neural network controller, then one can use policy gradient methods. The classic example of this is REINFORCE \cite{williams1992simple}, and all of it's modern incarnations, like Proximal Policy Optimization. 



% Existing examples are all limited to small systems, and tend to be fragile, dependent on hyper parameters, and prone to local minima. (need to cite all examples of this ! And also I'm not sure that
% s true).


These new differentiable simulators offer us analytic gradients through f, thus we can analytically use the update rule from \ref{eq:vapg}. We will refer to any technique that uses this update rule directly an Analytic Policy Gradient Method.

This may seem to be great news, and that we can finally use fast, gradient based methods for controller design. However, in practice these gradients have proven extremely challenging to use. Part of the problem is inherent to back propagation through time (BPTT), which is necessary to compute the gradient of the reward function. Long chains of BPTT have long been known to cause exploding or vanishing gradients, leading to difficulty in learning \cite{279181}. Recently Metz et. al. \citet{metz2021gradients} offer some exposition on the challenges of using the analytic gradients offered by these new rigid body simulators. That work highlights that they have found a common chaos based failure mode, which we will elaborate on later.


Yet another difficulty are severe local maxima. Local maxima are a common problem in all of deep learning, but it is apparent that using APG for reinforcement learning with rigid body systems is especially prone to falling into these extrema. The authors of Tiny Differentiable Simulator \cite{heiden2021neuralsim} agree, they used an optimization technique called Parallel Basin Hopping to circumvent this in the context of parameter identification.  

In this paper we propose one solution to the above difficulties. We propose an algorithm called Cross Entropy Analytic Policy Gradients, which combines APG with a zeroth order cross entropy based optimization method. We find that our method does significantly better than existing APG methods on a variety of (!!very specific pronlem). In fact we also outperform existing Model Free RL in these settings. 

\todo{I should go into more detail about the acrobot/pendulum, how normal RL struggles with it, how we are able to use it, talk about the role analytic gradients have to playz}


% Back in 1994 it was known that learning long term dependencies using neural networks is a difficult problem \cite{279181}. This is because Back Propogation Through Time (BPTT) causes exploding or vanishing gradients. One way to combat this is to introduct truncation. Another is to use a GRU or LSTM as the policy, which helps to control these gradients. 

% !! yeah definitley need some work here, the outline with everything else will help, need to make sure my thoughts are in order and make sense with what I do later. 

% % !! S
% It is known that existing model free deep RL struggles with the acrobot swing up and balance problem \cite{Gillen2020CombiningDR}. This is one area where our algorithm outperforms the existing deep RL algorithms. Need to flesh this out for sure. 


\section{Background And Related Work}


% Could definitely combine background and related work

\subsection{Deep Reinforcement Learning}

Deep Reinforcement Learning has seen a lot of attention and impressive results over the last decade or so, particularly in the context of continuous control for robotic systems. \cite{heess_emergence_2017} \cite{openai_learning_2018} \cite{lee_robust_2019} \cite{siekmann2021blind}. These problems are all high dimensional, nonlinear, underactuated, and they all involve complex contact sequences with the environments, which makes them very challenging for more traditional control design. 

DRL is usually divided into model based and model free control. Model based reinforcement learning, a classic example is PILCO \cite{deisenroth2011pilco}, uses predictions from a learned model of the environment to plan a trajectory. These approaches are typically much more sample efficient than model free RL, however model free RL often performs better in practice and does not require sometimes expensive re-planning at runtime.

Model free RL on the other hand learns a policy directly to maximize a reward function. This implies solving a more difficult optimization problem, but often works better in practice (citation needed). Examples of model free algorithms include  Soft Actor Critic (SAC), Proximal Policy Optimization (PPO), and Twin Delayed Deep Deterministic policy gradient (TD3) \cite{haarnoja2018soft} \cite{schulman2017proximal} \cite{fujimoto2018addressing}.

Our algorithm can be considered an on policy, model free reinforcement learning algorithm. It inherits many of their advantages and disadvantages of these algorithms. For example we can get good performance even on difficult, high dimensional tasks (\todo{can't really make this claim unless I solve a high dimensional task}). However our method is also relatively sample inefficient, and will require sim2real transfer to be applicable to physical systems (!! should maybe mention unique advantages of diff sims in this regard). 


% Traditional model-based control techniques are still very effective---arguably, Boston Dynamics still represents the state-of-the-art for legged locomotion in robotics, for example. However, these approaches require hundreds of expert person hours to develop each new controller. DRL attempts to automate at least some aspects of this challenging controller development process. There are already examples of learned policies outperforming ones hand-designed by experts~\cite{hwangbo_learning_2019}, and with the ever-continued growth and availability of computational power, there is good reason to believe these le

\subsection{Policy Search for Differentiable Physics}

Many of the papers which introduce a differentiable simulator also include a basic example using analytic gradients for policy design as a demonstration. Brax \cite{brax2021github} and the unnamed simulator developed by Degrave et. al. \cite{10.3389/fnbot.2019.00006} Both implement a basic version of APG. Brax's APG is used to command a fully actuated double pendulum to reach random targets, but they acknowledge that their APG is not able to solve most of the other problems in their benchmarking suite. Degrave Et. Al. manage to develop a walking gait for a quadruped, though it required a fairly significant amount of hand tweaking.

In \cite{qiao2021efficient} They introduce a simulator and suggest something called "policy enhancement" whereby they augment a model based RL algorithm with the analytic gradients to control a fully actuated double pendulum.

in \cite{pmlr-v139-mora21a} the authors present policy optimization via differentiable simulation (. They don't directly use the analytic policy gradient, and instead develop an indirect second order method. They also demonstrate their system on under-actuated systems like the inverted pendulum, the difference is that they are balancing at the stable equilibrium. This is a deceptively difficult problem, however we show that our method works for the swing-up case, which we argue is harder is harder. 

\subsection{Metahueristics}
\todo{What do i call this section?}


We were very inspired by basin hopping \cite{doi:10.1021/jp970984n} and the extension of parallel basin hopping \cite{doi:10.2514/6.2018-1452}. Tiny Differentiable Simulator \cite{heiden2021neuralsim} uses this method for parameter estimation in their own work. Our work differs obviously in the application, we are doing policy search.  

There are also several methods that combine a zeroth order optimizer with a local gradient based optimizer for robot learning \cite{pourchot2019cemrl} \cite{bharadhwaj2020modelpredictive} \cite{huang2021cemgd} \cite{pourchot2019cemrl}. However none of them do direct policy optimization, and none of them are using analytic gradients. 

%\cite{huang2021cemgd} CEM-GD
%     On the face of it, very similar, combining CEM with gradient descent, however they are in the context of trajectory planning, AKA model based RL. Ther
% \cite{bharadhwaj2020modelpredictive} (MPC ... not sure difference between this and CEM-GD)

% What about those that have tried analytic gradeints?



% How does our work fit in? It is similar to model based reinforcement learning (E.G. PILCO \cite{deisenroth2011pilco}) , however we are 

% What about model free?

% In my opinion our algorithms is closer to a model free algorithm, it stands to inherent any advantages of those methods, and can be used to augment them.

% The objective of model free reinforcement learning is to learn a policy directly in order to maximize some reward.. 


\begin{figure}[!htb]
 
    \includegraphics[width=.75\linewidth]{paper-template-latex/figs/rss_system_diagram.png}
    \caption{System Diagram \todo{Probably Don't Need this}}
  
\end{figure}

\subsection{Back Propagation Through Time}

In order to propagate gradients through an iterated system, we must use a technique called back propagation through time. It's a hard problem, as explained here \cite{279181}, and here \cite{metz2021gradients}. But there is no alternative, this is fundamental to the problem structure at hand. The issue is that our system is recurrent, the state at time t depends not just on the state action taken at time t-1, but on every action taken starting from the initial state. This means that in effect the chain of computations that results in a robots state becomes extremely long, which is the cause for much of the instability that we see. 

\begin{equation}
\frac{dR}{d\theta} = \frac{1}{T}\sum_{t=0}^{T}
\left[\frac{dr_{t}}{d\theta}\sum_{k=1}^{t} \frac{dr_{t}}{ds_{t}}\left( \prod_{i=k}^{t}\frac{ds_{i}}{ds_{i-1}}\right)\frac{ds_{k}}{d_{\theta}}\right]
\end{equation}

\todo{if I want to keep this equation I need to explain it ... }

What can be done about this? We will outline two techniques. The first is called truncated propagation through time. This in effect resets the gradient calculation every N steps. This is effective at stopping the gradient from exploding, and has been used effectively in the past at training RNNS. However we lose much of the long term dependence information, leading to short sighted policies. Furthermore in practice we have found that algorithms that depend on this parameter are very sensitive to it. 

We can also, either instead of or in addition to TBPTT, use specialized recurrent networks. Namely a GRU \cite{?} or LSTM. These networks were specifically designed with gates that allow for selectively "forgetting" information that flows through them, allowing for longer sequences to be learned from. Although in our case, we cannot replace our entire system with a GRU, only the controller, we have found in practice that using a GRU controller helped immensely in training analytic gradient based algorithms, so much so that we did not need to use truncated back propagation. 

\section{Problem Formulation}

\todo{I think this section needs a lot of work ... need to add the single cartpole if keeping it ... should I put in mass parameters , diagrams and the whole lot?  Maybe also move or repeat some of the RL stuff from the intro here??}

We consider two systems, a double cartpole pendulum, and an Acrobot \cite{spong_swing_1994}. It was demonstrated in \cite{Gillen2020CombiningDR} that most model free deep reinforcement learning struggles with the full version of the Acrobot in particular. It is worth elaborating on that point. Many benchmarking suites (for example, OpenAI's gym \cite{1606.01540}) have underactuated systems like the acrobot or cartpole pendulum, however the tasks are typically either to swing the system up, or to balance it, never the two together. In fact the one example of a suite with the swingup and balance task \cite{deepmindcontrolsuite2018} shows that the algorithms they tried on it didn't really work.  

For both systems, we use the variables $\phi_{1}$ and $\phi_{2}$ to refer to the joint angles. \todo{which is probably confusing, should I do a whole thing with a system diagram and like "we can write the dynamics in terms of the mass matrix etc etc" }


\subsection{Acrobat}
The Acrobot on the other hand was created by yours truly, it is more of a classical control system task, the only state varaibles are the joint angles and velocities, the reward is just the negative squared error. 

\[r_{a} = -\phi_{1}^{2} - \phi_{2}^{2} \]


\subsection{Inverted Double Pendulum}
\todo{This subsection either needs considerably more detail, or considerably less.. and I'm not sure which}

The pendulum environments are modified from Brax's existing benchmark environments, the only difference is that the initial condition is rotated 180 degrees from the upright. These environments are in the spirit of the RL community, and use some reward shaping, adding an an alive bonus as well as some feature extraction. rather than directly feeding in joint angles, both the reward and state variables are fed in as the x,y coordinate for the end of each link. The reward function for the environment is:


\begin{equation} 
r_{dp} = r_{alive} - r_{distance} - r_{velocity}
\end{equation}

Where 

\begin{equation}
r_{alive} = 10 
\end{equation}

\begin{equation}
r_{distance} = 0.05x^{2} + (y - y_{des})^{2}
\end{equation}

\begin{equation}
r_{velocity} = \dot \phi_{1} + \dot \phi_{2}
\end{equation}

and $y_{des}$ co-responds to the height of the second link when in the upright position


% \begin{equation}
% y = \sin(\phi_{1}) + \sin(\phi_{2})
% \end{equation}

% \begin{equation}
% x = x_{cart} + \cos(\phi_{2}) + \cos(\phi_{2}) 
% \end{equation}



% \[ r_{p} = r_{A} -\sqrt{\sin(\phi_{1})^{2} + \cos(\phi_{1})^{2}}\]

% Or in other words an alive bonus with a penalty for the euclidean distance of the tip of the pendulum to the goal state. The double pendulum likewise has the following reward: 

% \begin{equation} r_{dp} = r_{A} -(\sqrt{(\sin(\phi_{1}) + \sin(\phi_{2}))^{2} + (\cos(\phi_{1}) + \cos(\phi_{2}))^{2}}
% \end{equation}

Or in english: an alive bonus minus the euclidean distance between the end of the second link and the goal state. 

\subsection{The Brax Simulator}

We use Brax \cite{brax2021github} for all of our simulations. Brax is a differential physics engine that can simulate systems made up of rigid bodies, joint constraints, and actuators. This allows Brax to simulate a wide variety of robotic systems. Brax is built on top of the Jax \cite{jax} framework, and one of it's primary advantages is that it can run massively parallel simulations very quickly on accelerator hardware, I.E.  TPUs and GPUs. In addition to this, by virtue of being written entirely in Jax, we can take arbitrary gradients through the simulator using autodiff. This opens the door for our gradient based control method.  

% However as we've discussed, these analytic gradients suffer from some limitations that have made them difficult to use in practice. They have been found to have huge variance with respect to initial conditions. The appendix of the difftaichi \cite{hu2020difftaichi} paper has some good exposition on the subject. Many systems can have large "flat lands" in the reward space. (!!! offer billiard example here??, als, may move this discussion above when talking abpout gradients), and to large numbers of local minima. Another problem is that these gradients often have discontinuous jumps, imagine dropping a square object on the ground with a reward function equal to the final x position of the square. There will be a discontinuous jump with respect to the initial angle of the sqare (!!! this obviusly needs some work too). Yes another issue arises due to how autodiff handles conditionals. Although the auto-diff framework allows for gradients to propagate through conditionals, the auto-diff is essentially "blind" to these conditionals. For example, let's consider the classic cart-pole pendulum, which receives a reward of one for each step that the pole remains in the top half of the plane. The reward gradient for this system will always be zero. This obviously puts some limitations on the types of rewards suitable for this system. 

% All this is to say, there are problems with the gradients that can make learning hard. 



\subsection{Reward Landscape of the Acrobot}

In \cite{metz2021gradients} they show that the reward landscape of the Ant system in Brax. They show it is indeed, a mess. This is maybe to be expected for a system subject to hard contacts etc. One might expect the Acrobot, or inverted pendulum, or other contact-less systems to have much a much smoother reward landscape. We tested this assumption. We started with a random neural network policy, and then sampled a random vector from policy space. We then added used this vector to shift the initial policy, and recorded the sum of rewards that resulted. Note that we use the same initial condition for each trial, the only difference between rollouts is that a very slightly different policy is used. 


\begin{figure}[!htb]
        \centering
        \includegraphics[width=.8\linewidth]{paper-template-latex/figs/reward_landscape.png}
        \caption{Some projections of the reward landscape for the acrobot. We start with a random policy, and then sample a random vector from the space of policy weights. Each line represents the reward sum for one random direction through policy space
        \todo{Probably need to make the description clearer, and also I think it's an ugly plot}}
\end{figure}

As we can see, it's a mess, just an absolute mess, no wonder gradient descent isn't doing so hot. 



\section{Methods}

\todo{I think I should talk about APG first, then CEM, in the CEM section make it more concrete what we are doing (policy rollouts), and then how we combine these using }





% \subsection{Reinforcement Learning}

% The goal of reinforcement learning is to train an agent acting in an environment to maximize some reward function. At every timestep $t \in \mathbb{Z}$, the agent receives the current state $s_{t} \in \mathbb{R}^{n}$, uses that to compute an action $a_{t} \in \mathbb{R}^{b}$, and then receives the next state $s_{t+1}$, which is used to calculate a reward $r : \mathbb{R}^{n} \times \mathbb{R}^{m} \times \mathbb{R}^{n} \rightarrow \mathbb{R}$. We consider the finite time undiscounted reward case. The objective then is to find a policy  $\pi_{\theta}: \mathbb{R}^{n} \rightarrow \mathbb{R}^{m}$ such that:

% \begin{equation} \argmax_{\theta} \mathop{\mathbb{E}}_{\eta}\left[ \sum_{t=0}^{T}r(s_{t}, a_{t}, s_{t+1}) \right] \end{equation}
% where $\theta \in \mathbb{R}^{d}$ is a set that parameterizes the policy, and $\eta$ is a parameter representing the randomness in the environment. We will call the sum of rewards obtained during an episode a return. 


\subsection{The Cross Entropy Method}


The Cross Entropy Method (CEM \cite{RUBINSTEIN199789}) is a well established algorithm for importance sampling and optimization. Several other papers have used this method for control, although most of them use it as some sort of trajectory optimization (\todo{which papers, please cite}). We instead directly optimize the parametric policy, which is a more difficulty optimization problem, but which may be more general, and which is easy to deploy in real time after training is done. 

CEM maintains a probability distribution over its decision variables, in this case the decision variables are the parameters for our policy. The most common formulation is to use a normal Distribution $\mathcal{N}(\mu_{\pi}, \sigma_{\pi})$. At each step we sample from this distribution, and use the following update rules: 

\todo{Should I use another symbol... I guess $\sigma$ is usually standard deviation, either way, need to be consitent with notation here and in the algorithm section}

\begin{equation}
\label{eq:mean}
    \mu_{\pi}^{+} = \frac{1}{K_{e}}\sum_{i=0}^{K_{e}}\mu_{i}
\end{equation}


\begin{equation}
\label{eq:var}
    \sigma_{\pi}^{+} = \frac{1}{K_{e}}\sum_{i=0}^{K_{e}}(\mu_{\pi} - \mu_{i})(\mu_{\pi} - \mu_{i})^{T}
\end{equation}

However the covariance matrix grows quadratically with the number of policy parameters, and due to the size of our policies, which are sometimes many thousands of parameters, we make the following simplification to the variance:

\begin{equation}
\label{eq:var}
    \sigma_{\pi}^{+} = \frac{1}{K_{e}}\sum_{i=0}^{K_{e}}(\mu_{\pi} - \mu_{i})^{2}
\end{equation}

This implicitly ensures that our covariance "matrix" only has entries on the diagonal.

\subsection{Analytic policy gradients}


\todo{could add some more math here if I want}. 

We are interested in the gradient of the sum of the reward function for a given episode with respect to the parameters for our policy. Because these are directly available, we can easily apply Adam \todo{find the adam citation} or a similar gradient ascent optimizer to our problem. Thus our algorithm is as follows. Do a N rollouts with the current policy parameters, take the gradient of the mean of the sum of these roll outs, use that gradient to update the current policy, repeat until convergence. 



\subsection{Controller Architecture}

\todo{Can also add some math here}

To help combat the exploding / vanishing gradient problem that is inherent to iterated dynamical systems, we employ a Gated Recurrent Unit (GRU) network for our control policy \cite{8053243}. These networks were specifically designed to avoid these pathological gradient flows, and although on its own this change is not enough to obviate the exploding gradient problem, we found empirically that the difference helps dramatically. 

In addition to this, we use deterministic policies, rather than the stochastic ones usually associated with deep reinforcement learning. Typically, when one trains a policy, the policy function is typically outputting a probability distribution over possible actions at each timestep. At every time step then, one generates a new distribution based on the current state, and then samples from that distribution to select an action that is then applied to the environment. However we instead output the action directly. It is worth motivating this. Especially for an unstable system, changes in the random sampling can drastically alter the gradient \todo{theres a good example in the appendix to the difftaichi paper}. Therefore during a standard policy rollout, two rollouts with the same policy may have wildly different gradients, therefore requiring either many more samples to estimate the true gradient, or destabilizing training... at least that's my claim ... 


\subsection{Cross Entropy Analytic Policy Gradients}



\begin{algorithm}
\caption{Cross Entropy Analytic Policy Gradients}
\label{algo:ceapg}
\begin{algorithmic}
\Require Policy $\pi$ with trainable parameters $\theta$
\Require Hyper-parameters - $\sigma_{0}$, $K_{a}$, $K_{e}$
\State Sample $\theta_{c} = [\theta_{1}, ..., \theta_{n}]$ from $\mathcal{N}(\theta, \sigma^{2})^{K_{a} x \abs{\theta}}$
\For{$\theta_{i} \text{ in } \theta_{c}$}
    \State Run APG with initial policy weights $\theta_{i}$
    \State Collect sum of rewards $R_{i}$ and final policy $\theta_{i}$. 
\EndFor
\State \todo{Come up with notation for sorting by reward, also, be consistent with std vs variance}
\State $\theta^{+} = \frac{1}{K_{e}}\sum_{i=0}^{K_{e}}\theta_{i} $
\State $\sigma^{+} = \sqrt{\frac{1}{K_{e}}\sum_{i=0}^{K_{e}}(\theta - \theta_{i})^{2}}$


%\State $ \theta^{+} = \theta + \frac{\alpha}{n \sigma_{R}}\sum_{i=0}^{n} (R_{i} - R_{i+n})\delta_{i} $ 
\end{algorithmic}
\end{algorithm}


We combine these two algorithms as follows. Start with initial policy weight $\theta$, and an initial parameter variance $\sigma_{0}$. We then generate $K_{a}$ candidate policies by sampling from $\mathcal{N}(\theta, \sigma_{0})$. Using these policies as initial conditions, we run $K_{a}$ analytic policy gradient algorithms in parallel, which gives us new weight vector $ [ \theta_{0} , \theta_{1} ... \theta_{K_{a}} ]$, and the final rewards for these new policy weight, $R_{1}, R_{2} ... R_{n}$. We then order the policy weights in descending order based on their final return. Finally we select the top $K_{e}$ parameter weights and interpret them as the mean mean for new target distributions, and we use equations \ref{eq:mean} and \ref{eq:var} to update our parameter and variance vector. 




\section{Results}

\begin{figure*}[!h]
  \centering
  \begin{subfigure}[b]{0.32\linewidth}
    \includegraphics[width=\linewidth]{paper-template-latex/figs/pendulum.png}
  \end{subfigure}
  \begin{subfigure}[b]{0.32\linewidth}
  \includegraphics[width=\linewidth]{paper-template-latex/figs/acrobot.png}
  \end{subfigure}
  \begin{subfigure}[b]{0.32\linewidth}
      \includegraphics[width=\linewidth]{paper-template-latex/figs/double_pendulum.png}
  \end{subfigure}
  \caption{Reward curves for CE-APG}
  \label{fig:reward_curves}
\end{figure*}



We ran experiments on the aforementioned environments. For each experiment we ran trials with 8 random seeds. There are several sources of randomization during training, the inital value of the policy parameters, the noise added to the policy at the beginning of each iteration, and the initial condition of the simulator at the beginning of each episode. Figure \ref{fig:reward_curves}. For each of these trials we used the GRU controller architecture discussed above, with 2 hidden units of 64 hidden nodes each. During evaluation we performed roll-outs with the corresponding controller from N randomized initial conditions, and reported the results in table \ref{tab:results1}. In addition to our own algorithm, we also run comparisons from PPO, TD3, and Brax's own APG (which has some note-able difference from our own.) 


\todo{This paragraph probably belongs somewhere else?}
Differences between our APG and Brax APG. Different controller architectures, we use a deterministic GRU, they use a stochastic MLP. Related to this is the fact that the brax apg performs best when truncation is used in their BPTT algorithm, while we found this was not required. Furthermore the parellization characteristics are quite different, the Brax APG was designed for use with a TPU, and thus performs thousands of rollouts in parrell. Finally, they use the mean reward obtained from rollout out an environment with automatic resets. This implies they can't account for falls, sucks for them. More seriously though this doesn't effect the environments considered here, which are episodic but which do not implement early termination. 

It is also of note that we found the performance of CE-APG to be significantly faster on CPU compared to GPU/TPU, this is in stark contrast to other algorithms designed for brax, especailly the ones that do not make use 

Why compare to PPO. PPO is the "default algorithm" used these days by OpenAI, and indeed much of the rest of reinforcement learning community, also one of the most widely cited papers in the field. It is broadly applicable, relatively simple to implement.

Soft Actor Critic \cite{haarnoja2018soft} is an off policy reinforcement learning algorithm that performs well ona variety of tasks. We used it because it was found to work well relative to other DRL algorithms on the acrobot. \cite{Gillen2020CombiningDR}




\begin{table}[!htb]
\label{tab:results1}
\centering
\renewcommand{\arraystretch}{1.5}
\begin{tabular}{|l|l|l|l|}
\hline 
Environment   & Pendulum              & Acrobot               & Double Pendulum \\ \hline \hline
CE-APG (ours)    &   -630 $\pm$ 1000  &   -1961 $\pm$ 526       &   3410 $\pm$ 400\\  \hline
PPO               &    -540 $\pm$ 359 &    -4295 $\pm$ 35       &  -7.1e5 $\pm$ 1.9e6         \\  \hline
SAC               &    -253   $\pm$ 2    &     -3897.7 $\pm$ 718       &   1135 $\pm$ 851         \\ \hline
Brax Apg        &    -3079   $\pm$ 759    &     -2986 $\pm$ 818       &   -3179 $\pm$ 6.3e4     \\ 

\hline
\end{tabular}
\caption{Results of the training on our test environments, we report the mean and standard deviation of rewards obtained from training each algorithm with 8 different random seeds}
\end{table}



We can see that on the Acrobot and Double Pendulum, our algorithm significantly outperforms the others. We note that PPO gets extremely large negative rewards on the Double Pendulum environment. This is likely because there is no early termination in our environments, and velocities can get rather large during an episode with random actions (and recall, the reward includes a velocity squared term). Our algorithm does get edged out by SAC on the single link cartpole pendlum \todo{But only because of one bad seed that gets stuck at -3500, all the other seeds get the same -250 score as sac... }
What does -4000 vs -1500 actually look like, well, here's a sample from the acrobot.


\begin{figure}[!htb]
\captionsetup[subfigure]{labelformat=empty}
        \centering
        \subfloat[]{\label{sublable1}\includegraphics[width=0.6\linewidth]{paper-template-latex/figs/ce-apg_acrobot.png}} \\
        \subfloat[]{\label{sublable2}\includegraphics[width=0.6\linewidth]{paper-template-latex/figs/ppo_acrobot.png}}
        \caption{Best performing seeds for CE-APG vs PPO on the Acrobot. As we can see CE-APG stabalizes the system at the equilibrium, whereas \todo{Insert whichever algorithm} does not \todo{TODO make these one matplotlib plot to get rid of excessive whitespace, and make the legends line up etc}}
\end{figure}
% \begin{figure}[!htb]
 
%     \includegraphics[width=.6\linewidth]{paper-template-latex/figs/ce-apg_acrobot.png}
%     \caption{}
  
% \end{figure}

% \begin{figure}[!htb]
 
%     \includegraphics[width=.6\linewidth]{paper-template-latex/figs/ppo_acrobot.png}
%     \caption{}
  
% \end{figure}

As we can see, the CE-APG found a way to swing the system up into it's equilbruim state and keep it there, whereas the highest performing rival, despite achieving a decent reward, fails to the do same. 



% What about the double inverted pendulum? which included some reward shaping? Well the story is largely the same, let's look at figure 

% \begin{figure}[!htb]
% \captionsetup[subfigure]{labelformat=empty}
%         \centering
%         \subfloat[]{\label{sublable1}\includegraphics[width=0.6\linewidth]{paper-template-latex/figs/ce-apg_acrobot.png}} \\
%         \subfloat[]{\label{sublable2}\includegraphics[width=0.6\linewidth]{paper-template-latex/figs/ppo_acrobot.png}}
%         \caption{Best performing seeds for CE-APG vs PPO on the Acrobot. As we can see CE-APG stabalizes the system at the equilibrium, whereas PPO does not}
% \end{figure}





\section{Ablation Studies}

In addition to the comparisons done above, we also conducted several ablation studies. We conducted experiments using only the gradient free CEM method, with no analytic policy gradients being used. And also compared to only using our implementation of APG. For the APG we also add noise to the initial policy to give K=24 initial policies. We then run APG K times in parellel, but rather than periodically collapsing the policy using the CEM, we simply allow the gradient descent to continue. In both cases we still used the GRU controller architecture as before. In the case of CEM we used 2000 iterations for every environment, which we found to be well past the point of convergence. For APG we used 200*75 iterations, to give a comparable number of environment interactions to our CE-APG. The results are shown in table \ref{tab:ablation} \todo{not sure why this reference is broken}.

\begin{table}[!htb]
\label{tab:ablation}
\centering
\renewcommand{\arraystretch}{1.5}
\begin{tabular}{|l|l|l|l|}
\hline 
Environment         & Pendulum & Acrobot & Double Pendulum \\ \hline \hline
CE-APG (ours)    &   -630 $\pm$ 1000  &   -1961 $\pm$ 526       &   3410 $\pm$ 400 \\  \hline
PAPG             &    -2059 $\pm$ 1256       &     -2052 $\pm$ 457        &  1075 $\pm$ 449$^{*}$          \\  \hline
CEM              &    -1465 $\pm$ 287        &       -4473 $\pm$ 227       &   831 $\pm$ 243       \\ \hline

\hline
\end{tabular}
\caption{Results from ablation studies \todo{TODO explain that there were some NaNs in the double pendulum PAGP case}}

\end{table}


As we can see, the performance of either method individually is extremely poor, falling behind all the other methods together. \todo{Maybe I can add some reasoning as to why this would be}

\section{Discussion and future work}

    We have shown that the method introduced here is effective for a certain class of system, nonlinear underacted systems with unstable target states. This is interesting because other model free RL algorithms struggle with these systems. 

    This work also shows one way to use what turns out to be a very challenging tool effectively. It is still an open question as to what role analytic gradients have to play in learning based control of robotic systems in the future, however we think this paper shows that there is hope!
    
    Work remains in getting this algorithm to work effectively in contact rich environments. There is active work in the community to make brax friendlier to analytic gradient based algorithms, in particular adding soft contacts, which may very well help a lot. 
    
    Other improvements could be made, for example importance sampling has been found to improve the sample efficiency of CEM by a factor of 10. We were not able to effectively implement this improvement due to some technical limitations with pmap. Basically we are constrained to always use the same number of threads during each round of computation. Also we could increase the value of $K_{a}$ if only I wasn't so lazy and vmapped my pmap... 

\section{Conclusion} 
\label{sec:conclusion}

In conclusion, we presented Cross Entropy Analytic Policy Gradients, an algorithm that can exploit analytic gradients. We covered some relevant background, introduced our method, and placed it in the broader context. We then presented our algorithm and the environments we used for testing. We then presented results of our algorithm compared to state of the art baselines, as well as some ablation studies. We dug into detail with the acrobat environment, contextualizing the rewards we presented. This demonstrated that our algorithm was able to successfully stabilize the system, whereas our competitors where not. We think this algorithm is exciting and can be built upon. 
%\section*{Acknowledgments}


%% Use plainnat to work nicely with natbib. 

\section{Apendix}
\subsection{Hyper Params}

We report the hyper parameters used for each experiment here. In all cases we used a manual hyper parameter search. 


\begin{center}
\begin{tabu}{ X[2,l] | X[1,l] }
CE-APG Hyperparameter & Value \\
 \hline
 Episode length ($N_{e}$) & 500  \\ 
 Action Repeat & 1  \\ 
 APG Epochs & 75 \\ 
 Total Epochs & 200 \\ 
 initial std & 1e-3 \\
 learning rate & 5e-4 \\
 $K_{a}$ & 24 \\
 $K_{e}$ & 8 \\
\end{tabu}
\end{center}


\begin{center}
\begin{tabu}{ X[2,l] | X[1,l] }
PPO Hyperparameter & Value \\
 \hline
 Episode length ($N_{e}$) & 500  \\ 
 Action Repeat & 1  \\ 
 Total Timesteps & 8e7 \\ 
 Reward Scaling & 1 \\
 Minibatch Size & 32 \\
 Batch Size & 512 \\
 Unroll Length  & 50 \\
 N Update Epochs & 8 \\
 Discounting & 0.99 \\
 Learning Rate & 3e-4 \\
 Entropy Cost & 1e-3 \\
 N Envs & 512 \\
\end{tabu}
\end{center}


\begin{center}
\begin{tabu}{ X[2,l] | X[1,l] }
SAC Hyperparameter \todo{TODO} & Value \\
 \hline
 Episode length ($N_{e}$) & 500  \\ 
 Action Repeat & 1  \\ 
 Total Timesteps & 8e7 \\ 
 Reward Scaling & 1 \\
 Minibatch Size & 32 \\
 Batch Size & 512 \\
 Unroll Length  & 50 \\
 N Update Epochs & 8 \\
 Discounting & 0.99 \\
 Learning Rate & 3e-4 \\
 Entropy Cost & 1e-3 \\
 N Envs & 512 \\
\end{tabu}
\end{center}



\begin{center}
\begin{tabu}{ X[2,l] | X[1,l] }
Brax-APG Hyperparameter \todo{TODO} & Value \\
 \hline
 Episode length ($N_{e}$) & 500  \\ 
 Action Repeat & 1  \\ 
 APG Epochs & 75 \\ 
 Total Epochs & 200 \\ 
 initial std & 1e-3 \\
 learning rate & 5e-4 \\
 $K_{a}$ & 24 \\
 $K_{e}$ & 8 \\
\end{tabu}
\end{center}




\chapter{Conclusions}



%=== Appendix ============================================
\appendix

\dsp

\chapter{Appendix Title }{\label{appendix:a}}
\begin{section}{Section Title}

Appendicitis

\end{section}
\end{mainmatter}

%----- Bibliography ----------------
\ssp
\bibliographystyle{JHEP3}
\bibliography{dissertation}

\end{document} 
